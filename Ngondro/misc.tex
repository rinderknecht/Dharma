\newpage

\begin{center}
Libre glose inspir�e de Mipham Namgyal Rinpoche (1846-1912)\\
selon la Grande Perfection (tib. \emph{dzogchen},
skt. \emph{Mah\={a}sa\.{n}dhi})
\end{center}

\bigskip\bigskip

\begin{verse} 
\lit{orgyen y�l gyi nub jang tsham}\\
{\small Contemplant le ciel sans nuages, la clart� latente du c{\oe}ur
  se d�ploie}\\
\lit{pema gesar dongpo la}\\
{\small tel un brocart marine brod� de perles argent�es. Un
  rougeoiement cr�pusculaire irise alors la nacre en cocardes
  quinticolores.}\\ 
\lit{yatshen chog gi ng�drub nye}\\
{\small 
 Par la stabilit�, les quatre visions merveilleuses s'�panouissent
 successivement, culminant avec leur dissolution, qui est ins�parable
 de la  r�sorption de l'esprit du pratiquant en sa nature
 fonci�re.}\\
\lit{pema jungne shesu drag}\\
{\small Ce fruit est connu comme �~Atteinte de l'�veil primordial.~�}\\
\lit{khor du khandro mangp� kor}\\
{\small Puis, de la clart� de la sagesse spontan�ment pr�sente,
  �manent � nouveau rais et sph�res armillaires quinticolores
se mouvant librement  dans l'espace.}\\
\lit{khye-kyi jesu dag drub}\\
{\small En demeurant dans la contemplation de l'esprit naturel, pur et
  sans origine, les quatre �l�ments se r�sorbent en aur�oles.}\\
\lit{jin gyi lab chir sheg su s�l}\\
{\small Afin d'atteindre le corps d'arc-en-ciel pour le bien d'autrui,
puiss�-je purifier tous les ph�nom�nes en la sagesse primordiale!}
\end{verse}
