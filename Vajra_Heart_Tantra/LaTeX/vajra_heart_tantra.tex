\documentclass[a4paper,11pt,twoside,final]{book}

% Page geometry for College Publications
%
\usepackage[bindingoffset=0cm,a4paper,centering,textheight=200mm,textwidth=120mm,includefoot,includehead,dvips]{geometry}

% Language and fonts
%
\usepackage[british]{babel}    % British English
\usepackage[T1]{fontenc}       % Required for hyphenation and \DJ
\usepackage[utf8]{inputenc}    % UTF-8 encoding
%\usepackage{hyphenat}          % \hyp{} is a breakable dash
%\usepackage{url}               % To typeset URLs
\usepackage{mdwlist}           % Tight vertical spacing of list items
%\usepackage[scaled]{luximono}  % Nice bold/italic teletype font
\usepackage{microtype}         % Microtypographic enhancements

% Bibliographic style
%
%\usepackage[comma]{natbib}
%\bibliographystyle{plainnat}
%\usepackage{etoolbox}
%\apptocmd{\thebibliography}{\raggedright}{}{} % No Underfull \hbox

% Miscellanea
%
%\usepackage{xspace}    % Automatic insertion of a space after a macro
%\usepackage{varioref}  % Qualifying references with page numbers
%\usepackage{booktabs}

% Title and author
%
\title{{\Huge The Vajra Heart Tantra}\\
A Tantra Naturally Arisen from the Nature of Existence
from the Matrix of Primordial Awareness of Pure Perception}
\author{\Large Dudjom Lingpa\footnote{Copyright Gyatrul Rinpoche \&
    B.~Alan Wallace. All rights reserved. No copies of this manuscript
    are permitted without the written permission of the translator.}}
\date{}

% ---------------------------------------------------------------------
%
\begin{document}

\maketitle

\begin{center}
  Translated from the original Tibetan under the guidance of Gyatrul
  Rinpoche by B.~Alan Wallace.
\end{center}

\chapter{Preface}

The text translated here as \emph{The Diamond Heart Tantra: A Tantra
  Naturally Arisen from the Nature of Existence from the Matrix of
  Primordial Awareness of Pure Perception}\footnote{Tibetan title: Dag
  snang ye shes drva pa las gnas lugs rang byung gi rgyud rdo rje'i
  snying po. Sanskrit title:
  Vajrahrdayasuddhadhutijnanaharesrilamjatiyatisma. Collected Works of
  H.H. Dudjom Rinpoche. The following concise biography of Dudjom
  Lingpa is drawn from the following texts: Tulku Thondup, The Origin
  of Buddhism in Tibet: The Tantric Tradition of the Nyingmapa
  (Marion, MA: Buddhayana, \oldstylenums{1984}); Tulku Thondup
  Rinpoche, Hidden Teachings of Tibet: An Explanation of the Terma
  Tradition of the Nyingma School of Buddhism, ed. Harold Talbot
  (London: Wisdom, \oldstylenums{1986}); Dudjom Lingpa, Buddhahood
  Without Meditation: A Visionary Account Known as Refining Apparent
  Phenomena (Nang-jang), trans. from the Tibetan under the direction
  of Chagdud Tulku Rinpoche by Richard Burton (Junction City, CA:
  Padma Publishing Co., \oldstylenums{1994}; Dudjom Rinpoche, The
  Nyingma School of Tibetan Buddhism, trans. and ed. by Gyurme Dorje
  with Matthew Kapstein (Boston: Wisdom, \oldstylenums{1991}); and
  H.H. Dudjom Rinpoche Jigdral Yeshe Dolje, Dudjom Tersar Ngondro (New
  York: Yeshe Melong, \oldstylenums{1992}).}  is among the principle
``mind treasures'' of Dudjom Lingpa
(\oldstylenums{1835}--\oldstylenums{1904}), one of the great Dzogchen
masters of the Nyingma order of Tibetan Buddhism in recent
history. Commonly known among Tibetans as \emph{Neyluk Rangjung}
(Naturally Arisen from the Nature of Existence), it is regarded as a
general commentary to another of Dudjom Lingpa's great works,
\emph{Buddhahood Without Meditation: A Visionary Account Known as
  Refining Apparent Phenomena}, which has been beautifully translated
into English by Richard Burton under the guidance of Chagdud Tulku
Rinpoche.\footnote{Dudjom Lingpa, Buddhahood Without Meditation: A
  Visionary Account Known as Refining Apparent Phenomena (Nang-jang),
  trans. from the Tibetan under the direction of Chagdud Tulku
  Rinpoche by Richard Burton (Junction City, CA: Padma Publishing Co.,
  \oldstylenums{1994}.} Also known as Garwang Dudjom Pawo, Dudjom
Lingpa was born in the Golok region of eastern Tibet on the tenth day
of the first month of the Sheep Year. According to a number of ancient
and recent prophecies, as written by one of his own subsequent
emanations, H.H. Dudjom Rinpoche, his previous incarnations include
the following masters of the Buddhist tradition:
\begin{enumerate*}

  \item Nuden Dolje Chang: the buddha who bestowed empowerments upon all
    the thousand buddhas of this fortunate eon;

  \item Sariputra;

  \item Saraja;

  \item Krsnadhara;

  \item Humkara;

  \item Drogpen Kyeuchung Lotsa;

  \item Smrtijnana;

  \item Rongzom Chokyi Zangpo;

  \item Dampa Deshek;

  \item Lingje Repa;

  \item Chogyal Pagpa;

  \item Drumgyi Karnagpa;

  \item Hepa Chojung;

  \item Tragtung Duddul Dorje;

  \item Sonam Deutsen;

  \item Duddul Rolpa Tsel.

\end{enumerate*}
According to Dudjom Lingpa's own autobiography,\footnote{The
  Autobiography of the bDud jom-ging pa (gTer chen chos kyi rgyal po
  khrag 'thung bdud 'joms gling pa'i rnam par thar pa zhal gsung ma),
  by Dud joms Ling pa, ed by Padma-lung rtogs rgya-mtsho (Dehradun:
  G.T.K. Lodoy and N. Gyaltsan, \oldstylenums{1970}).} during the
first three years of his life, he saw hosts of dakinis and protective
deities looking over him.  On one occasion a dakini led him to
Oddiyana, the realm of the dakinis, where he encountered Vajravarahi,
foremost of the dakinis, from whom he received great blessings. During
his youth, he also spent one human day at Palri on the subcontinent of
Ngayab (Skt.?), which equalled twelve years in the time of that
realm. There he received teachings from Padmasambhava himself. On
various occasions he also received prophecies from Padmasambhava's
consort Yeshe Tsogyal, who she cared for him her like own son.

Dudjom Lingpa had eight renowned sons, including Jigme Tenpey Nyima,
the Third Dodrup Chen Rinpoche
(\oldstylenums{1865}--\oldstylenums{1926}), who was a consummate
scholar and adept of both the sutras and tantras. During the course of
his life, Dudjom Lingpa performed many miracles, and he reached the
highest stages of realization of the stages of generation and
completion as well as the Great Perfection. It is said that thirteen
of his disciples attained the Rainbow Body, and one thousand became
vidyadharas through gaining insight into the essential nature of
awareness.

Dudjom Lingpa's subsequent incarnations include His Holiness Dudjom
Rinpoche, Jigdral Yeshe Dorje, his mind emanation and the editor of
the Tibetan version of the text translated here; Tulku Kunzang Nyima
(Tib. sPrul sku kun bzang nyi ma) his speech emanation; Sonam Detsen,
his body emanation; Jamyang Natsok Rangdrol, also known as Dorje
Dragtsel Lingpa, an emanation of his enlightened activities; and Tulku
Drachen (Rahula, sp?), an emanation of his enlightened qualities.

In one of Dudjom Lingpa's dreams, a devaputra named Dunggi Zur\-phu
prophesied that the benefit from his profound hidden treasures would
go west, saying, ``Those deserving to be tamed by you dwell in human
cities to the west.'' In the hope to help fulfill that prophesy, the
present translation of his hidden treasure The Diamond Heart Tantra
was made under the guidance of the Venerable Gyatrul Rinpoche, who has
been teaching in the west since \oldstylenums{1972}. Gyatrul Rinpoche
received the oral transmission of this text three times from three of
the emanations of Dudjom Lingpa: in Tibet he received it from Jamyang
Natsok Rangdrol and from Tulku Kunzang Nyima, and later in Nepal he
received it from His Holiness Dudjom Rinpoche, Jigdral Yeshe Dorje,
the Supreme Head of the Nyingma order of Tibetan Buddhism.

Beginning in the autumn of \oldstylenums{1995}, I read through this
text once with Gyatrul Rinpoche, receiving many points of
clarification from him. While working on a first-draft translation, he
went through the text with me line-by-line, carefully correcting
errors in my translation and elucidating points of lingering
uncertainty in my own comprehension of the text. I am deeply grateful
to him for opening this treasure to me for the benefit of all those
who may read our translation. I am also indebted to Dr.~Yeshi Dhonden
and Khenpo Tsewang Gyatso for elucidating some points of the text.

\chapter{Introduction}

Homage to the manifest face of Samantabhadra himself, the omnipresent
Lord, the original, primordial ground. Because the minds of all the
jinas of the three times are of one taste in the absolute nature of
reality, there is the Mind Lineage of the jinas. Because the symbolic
signs of the space treasury of reality-itself are spontaneously
released without reliance upon the stages of spiritual training and
practice, there is the Symbolic Lineage of the Vidyadharas. Because
the practical instructions naturally arise as an oral lineage and an
entrance to the disciples' path, in the manner of filling a vase,
revealed by themselves, not by human beings, as an apparitional
display of primordial wisdom, there is the Oral Lineage of
Individuals. To set forth the great reality in which these three are
unified as the spiritual mentor of the world, bless me with the circle
of ornaments of the inexhaustible three mysteries of the jinas and
jinaputras. The myriad of the oath-bound three roots have granted
their permission, so please allow this composition to be brought to
completion.  Having concealed the original, primordially pure nature
of existence, great intellect-transcending reality-itself, free of
conceptual elaboration, by the obscuration of self-concepts and
grasping onto duality, individuals are bound by clinging onto the true
existence of the deceptive cycle of the three realms.  However, there
are those who have accumulated vast merit over many eons and are
imbued with the power of pure prayers. By arousing the karmic momentum
of engaging in the deed of nonaction in great, self-arisen, primordial
wisdom, for the sake of those fortunate ones who are empowered with
respect to the space-treasury of reality-itself, I shall present this
king of tantras, a fundamental tantra that has naturally arisen from
the nature of being of the sugatagarbha.

Here is the way this tantra first arose. On the evening of the
fifteenth day of the first lunar month of the Male Water Dog Year, due
to the power of the profound, swift path of the Leap-over, the direct
vision of reality-itself appeared; and as a result of a little
practice of the path of the method of the stage of generation, I
reached the ground of a fully matured vidyadhara. By the power of
arriving there, all appearances and mental states dissolved into
primordially pure reality-itself, the absolute nature free of
conceptual elaboration. Then the very face of the dharmakaya became
manifest. After some time, spontaneous appearances arose like this in
the form of a buddha-field: on that very occasion of self-arisen,
primordially pure, great bliss, my environment was the naturally
appearing, real Akanista. This apparitional buddha-field was vast and
spacious, with its surface smooth and level, and pliant to the
touch. Grassy mountains of medicinal herbs were fragrant with mists of
fine aromas. The whole ground was completely covered with various
radiant, luminous, clear, sparkling, shimmering, lovely flowers in
shades of white, yellow, red, green, blue, and multiple hues. In the
four directions were four oceans of ambrosia imbued with eight
excellent qualities. On the shores of those great oceans were pebbles
of jewels, sand of gold, turquoise meadows, and overarching halos of
rainbows. Forests of wish-fulfilling trees flourished in the four
cardinal directions, emanating billowing clouds of sensory
offerings. Various types of beautiful, emanated birds sang forth the
sounds of Dharma with gentle, soothing calls. Various lovely, emanated
animals frolicked about and appeared to be contentedly listening to
the Dharma. The whole sky was covered with checkered patterns of
lattices of rainbow light. Everywhere the sky was filled with singing
and dancing viras and viras, and many goddesses made sensory offerings
and devotions.

In the center of that region, in a great, delightful garden, resting
against a tree covered with foliage and flowers, there was a vast and
lofty jeweled throne supported by eight lions. The branches of the
tree were draped with various silk hangings, and full and
half-garlands of jewels, and many tinkling peppercorn bells gave rise
to the natural sounds of the holy Dharma. Upon that lion-throne was a
lotus, sun, and moon seat, upon which was the true Teacher
Samantabhadra Saroruhavajra, naturally appearing with the radiance of
the ground of reality. His body was blue-black in color, bearing the
features of an eight-year-old youth. His right hand displayed the
mudra of expounding the Dharma, and his left hand was in the mudra of
meditative equipoise. He was adorned with the signs and symbols of
Awakening and all the apparel of a sambhogakaya. Within the realm of
his ocean-like, limpid, transparent body, naturally appeared all the
peaceful and wrathful jinas and myriads of buddha fields and
emanations, like bright reflections of planets and stars in a lake.
Incalculable rays of blazing light emanated from him, and from their
tips appeared various symbolic letters.

Gathered around the Teacher were natural displays of the entire
assembly of his apparitional retinue of
\oldstylenums{84},\oldstylenums{000} disciples, including Bodhisattva
Vidyavajra, Bodhisattva Prajnendra, Bodhisattva Jnanavajra,
Bodhisattva Mahasahasrananta, Bodhisattva Prasannatindra, Bodhisattva
Anabhogavyuha, Bodhisattva Parabhasa Vasavartin, Bodhisattva
Aksendriya, Bodhisattva Srotendriya, Bodhisattva Ghranendriya,
Bodhisattva Jihvendriya, and Bodhisattva Kayendriya. They all were
gazing at the face of the Teacher, and they remained silent as they
reverently bowed. The Teacher remained silent as he looked into the
realm of space. At that time, these natural sounds of reality itself
emerged from the realm of pristine space: Ah. The whole of samsara and
nirvana is groundless and rootless.  The Vajra Queen is great space.
The great emptiness of space is the Great Mother.  All phenomena are
apparitions of reality-itself and the sole nature.  Everything is born
from the unborn.  The emergence of arising apparitions ceases.  Causes
and conditions are extinguished right where they are.  Thus, the
Teacher and the Teaching, the path and its fruits, in reality-itself
are devoid of signs and words.  The many avenues of method and wisdom
appear as the great, natural event and natural arising.  The space of
no-object and great openness is limpid, clear, and free of
contamination.  All displays of the buddha-field, Teacher, and retinue
are nonexistent, but from nonexistence they appear as existent.  That
we praise with great wonder!  As soon as those sounds occurred, the
entire assembled retinue said to the Bhagavan with one voice, ``O
Teacher, Bhagavan, Pervasive Lord, and Immutable Sovereign, please
listen and heed our words! May the Teacher explain why this entire
buddha-field, the Teacher, and the assembled disciples are here and
tells us whence this all arose.''

The Teacher replied, ``O apparitional disciples, listen! Why have
these apparitions of primordial wisdom --~the buddha-field, Teacher,
and disciples~-- arisen? They are for the sake of revealing an
entrance to the nonconceptual primordial wisdom of the minds of all
the sugatas of the three times in accordance with the faculties of all
sentient beings who are wandering in the cycle of existence of the
three realms. The vision of the naturally appearing, real Akanista as
a buddha-field will reveal the great vision of reality-itself by the
great power of wisdom and primordial wisdom. I am the primordial
ground, the Teacher who naturally appears from the innate luster of
the sugatagarbha, appearing to myself. The innate luster of empty
awareness, free of conceptual elaboration, appears as Vidyavajra. The
innate luster of the wisdom of identitylessness appears as
Prajnendra. The innate luster of the eight collections of
consciousness, together with the mental factors, appears as the
assembled retinue.''

\section{The Questions of Prajnendra}

Then Bodhisattva Prajnendra rose from his seat and asked the Bhagavan,
``O Teacher, Bhagavan, if you appear as the innate luster of the
sugatagarbha, if I, Prajnendra, appear as the innate luster of wisdom,
if Vidyavajra appears as the innate luster of awareness, and if the
assembly of male and female bodhisattvas appears from the eight
collections of consciousness, together with the mental factors, we
should appear in that way to all the sentient beings of the three
realms. However, in the midst of delusive appearances of joys and
sorrows, friends and enemies, they continue in the three realms of
existence, where miseries occur and pure appearances do not. Why is
that? May the Teacher explain!''

He replied, ``O son of good breeding, sentient beings who have been
reduced to the ethically neutral ground do not see pure
appearances. Impure, delusive mental states and appearances concerning
friends and enemies and joys and sorrows are characteristic of
ordinary sentient beings. Here inconceivable pure appearances arise
due to the great power of wisdom and primordial wisdom to individuals
who have previously sat in the presence of the non-human, naturally
appearing Teacher, the perfect Buddha, and Orgyen Padmavajra. Those
people attained the supreme siddhi after entering the gateway of
Vajrayana Dharma and applying themselves diligently to its
practice. Then, until the myriads of realms of sentient beings are
empty, they bring forth the power of pure prayers, as a result of
which they repeatedly arise as Teachers for the sake of the world,
teaching in accordance with the individual needs of disciples. Due to
the stimulation of their previous karmic momentum, they directly see
the truth of reality-itself, and they emerge from the realm of
wisdom. To them these pure appearances arise, but they are not minds
and mental factors. Rather, these appearances are by nature creative
displays of manifest, absolute reality. They are not the eight
collections of consciousness, but they are not anything else; so they
are called by those names. They appear in numerous ways from the
nondual Teacher and retinue. Those known as bodhisattvas have gone
well beyond mundane existence, even though they have not become
buddhas; and so they are known.''

Again Bodhisattva Prajnendra asked, ``O Teacher, Bhagavan, so it
is. Thus, if the apparitional Teacher and the entire retinue are
nondual and not different, there is no point to all the teaching and
listening on the part of the Teacher and the assembly of
disciples. There is no difference in the quality of everyone's
primordial wisdom, so what point is there in putting on the show of
teaching and listening? May the Teacher explain!''  He replied, ``O
Prajnendra, in the past the naturally appearing Teacher of disciples
known as Sakyamuni, a Teacher like the sun, arose as an emanation like
sunlight. The Teacher, those who requested [the Dharma], and the
retinue acting as listeners appeared to be teaching and listening to
individual sorts of spiritual paths and vehicles for training
disciples. Although the Teacher and the retinue were nondual, for the
sake of the disciples, various expressions of skillful means were
displayed, like an illusionist and his illusion.''

\section{The Questions of Mahasahasrananta}

Then Bodhisattva Mahasahasrananta reverently bowed to the Bhagavan,
and with his palms pressed together he requested, ``O Teacher,
Bhagavan, so that all sentient beings may be liberated from the ocean
of miseries of mundane existence and reach the state of liberation,
please grant us profound, practical instructions for manifestly
achieving the state of the fully perfected Buddha Samantabhadra in one
life and with one body.'' 

The Teacher replied, ``O all you assembled disciples, including
Mahasahasrananta, listen! The great, sublime path that brings all
sentient beings to the grounds and paths of liberation is called the
swift path of the clear light Great Perfection. This is the most
sublime of all Dharmas, a general synthesis of all paths, the
destination of all spiritual vehicles, and an expansive treasury of
all secret mantras. However, the only people who will encounter this
path will be those who have stored vast collections of merit in many
ways over incalculable eons, who have repeatedly made extensive
prayers for the state of perfect Awakening, and who have previously
sought the path and established habitual propensities for it in order
to reach the path by means of other vehicles. No one else will do
so. Why? Under the influence of negative deeds and due to the strength
of powerful, devious maras of mental afflictions, even if unfortunate
people are present where this vehicle is being explained and heard,
their minds will be in a wilderness five hundred yojanas away.  Those
unfortunate servants of mara, with their perverse aspirations, act
contrary to this profound Dharma and respond to it with abuse, false
conjecture, repudiation, jealousy, and so on.  On the other hand,
those who enter this gateway and implement the meaning [of these
teachings] will never be more common than the appearance of a star
during the daytime. Upon entering [this path], some will dispense with
it after acquiring mere understanding and learning, and they will go
astray with no sense of conscientiousness. Not engaging in spiritual
practice, they will face death as ordinary beings, and they will not
achieve liberation.  In general, to enter this vehicle and put it into
practice, one must have all the following characteristics:
\begin{itemize*}

  \item belief in the Dharma and one's spiritual mentor;

  \item trust in the path without uncertainty;

  \item earnest mindfulness of death and recognition of composite
    phenomena as being impermanent, thus having little attraction to
    mundane activities;

  \item contentment with respect to food, wealth, and enjoyments;

  \item insatiability towards Dharma due to great zeal and
    determination;

  \item unifying one's life and spiritual practice without
    complaining.

\end{itemize*}
If people with mental stability who are not boastful about the mere
number of months and years they have spent practicing in retreat see
this entrance and practice, they will undoubtedly achieve the supreme
state of the Jina Vajradhara in this very lifetime. It is said of
other vehicles that after accumulating [merit and knowledge] and
purifying obscurations for three countless eons, one finally,
manifestly becomes perfectly enlightened. However, under the influence
of karma, mental afflictions, and habitual propensities over many
eons, while one continues through many lifetimes, one is influenced by
various types of thoughts and deeds, making it difficult to meet with
the path of accumulation and purification. Think carefully about this
situation, and you will come to a clear certainty.  Be that as it may,
at the present time, due to excellent karmic connections from the
past, you have obtained a sublime human life with leisure opportunity,
and you have encountered the most sublime of Dharmas, the Secret
Mantra Vajrayana. This is no time for ultimately holding onto the hope
of accumulating [merit and knowledge] for a long time and then finally
attaining Awakening. Rather, you must apprehend the ground of your own
being for yourself by experiencing the self-nature of the
sugatagarbha, the primordial ground that is the path to liberation in
this life. Apart from that, the teachings concerning the possibility
of achieving the state of liberation as a result of accumulating much
karma from one life to another are effective for bringing about
temporary happiness in the minds of sentient beings; but buddhahood by
that means is extremely difficult. Consider that those teachings may
have merely a provisional meaning.''

Bodhisattva Mahasahasrananta commented, ``O Teacher, Bhagavan, in this
present human life one may strive for good, mind-expanding attitudes
and for bodily and verbal virtues. Then by practicing at some time in
the future the view and meditation of the clear light Great
Perfection, the vajra essence of the Secret Mantra, one might possibly
achieve liberation. But it is said that it is difficult to achieve
liberation merely by practicing in this life. Moreover, it is said
that small minded beings, such as sravakas and pratyekabuddhas, cannot
fathom the deep and vast Vajrayana Dharma. Is that true or not? If it
is true, while there may be differences in quality and capacity in the
minds of sentient beings, if it were necessary to transform into a
being of the Mahayana class, the small-minded would have to become
greater, and there would have to be those whose minds are so immense
that they cannot expand. If that were so, one would have to acquire
some Secret Mantra from somewhere other than one's own mind-stream. In
that case, I don't know what that would mean. May the Teacher
explain!''

He replied, ``O Mahasahasrananta, if you think like that, if you have
arrived at the gateway of the Secret Mantra in this present lifetime,
you have firm faith of belief in that, and you have strong, unflagging
enthusiasm, the time has come to practice. When fortunate people
encounter the gateway of the profound Secret Mantra, apart from simply
achieving firm faith of belief, they will never have any other
clairvoyance, omens, or auspicious circumstances to make them think
that now the time has come to practice Secret Mantra. Once you have
obtained a human life and encountered the Secret Mantra Dharma and a
spiritual mentor, if that is not the time to practice the Great
Perfection, there will certainly be no other auspicious circumstance
than that in any other lifetime. It is not that sravakas,
pratyekabuddhas, ordinary beings and so forth are too
small-minded. Rather, due to their previous karma, they do not
encounter the gateway of the Secret Mantra. Or even if they do, they
have no faith and no belief, and as a result of spiritual sloth and
distraction, they do not practice.  Know that this has nothing to do
with the specific capacities of people's minds. Do not think that
there are differences in the capacities of the minds of all sentient
beings. To those fettered by selfishness, I teach that by opening
their hearts to all sentient beings throughout space, without concern
for their own self-interest, they will see the reality of the
nonduality of self and other.''

Mahasahasrananta continued, ``O Teacher, Bhagavan, if that is the
case, is it impossible for them to expand their minds by meditating on
the profound mystery of the Great Perfection? Or, even if they
meditate on the Great Perfection, must they bring forth the Spirit of
Awakening by some other means? May the Teacher explain!''

He replied, ``O son of good breeding, this Great Perfection is the
vehicle to unsurpassed fruition. The realization of that great reality
that pervades the whole of samsara and nirvana is called the ultimate
Spirit of Awakening of the ground. Just that is to be
apprehended. Apart from that, the so-called Spirit of Awakening that
is strenuously fabricated with the intellect is a mind-state in which
one views oneself as the meditator and other sentient beings as
objects of meditation. That attitude is as confined as a teacup. In
the realm of the Great Perfection, the original nature of the great
equality of samsara and nirvana, the mode of existence of the
ground-itself is known just as it is by means of great, all-knowing,
primordial wisdom of omniscience. To speak of having a Spirit of
Awakening greater than the vision of great, all-seeing primordial
wisdom would be like having water and yet saying you must seek liquid
elsewhere. The original, primordially pure ground, the great reality
that pervades the whole of samsara and nirvana, is the Spirit of
Awakening. Without that knowledge, just the kind sense of compassion
and love of parents for their children is an objectifying, conceptual
state of mind.  With that alone, one might aspire for fortunate
rebirth, but hoping that this will lead to buddhahood is as senseless
as hoping that the son of a barren woman might become a householder.''

Bodhisattva Mahasahasrananta requested, ``O Teacher, Bhagavan, may the
Teacher reveal the profound path that liberates disciples!'' 

He replied, ``O son of good breeding, there appear to be many avenues
of skillful means and wisdom that serve as entrances to the city of
great liberation. But ultimately taking the mind as the path is the
quest for the true path; and once the ground has been determined, one
may take reality-itself as the path. Between those two alternatives,
first of all here is the way to take the mind as the path.  At the
outset, disciples who maintain their samayas initially train their
minds by way of the common, outer preliminaries, namely the Four
Thoughts that Turn the Mind, and the uncommon, seven inner
preliminaries. At the conclusion of those, here is the way to engage
in the stages of the main practice of seeking the path.  First of all
go to a totally secluded forest, offer prayers of supplication to your
spiritual mentor, merge your mind with that of your mentor, then relax
for a little while. O Mahasahasrananta, among your body, speech, and
mind, which is most important? Who is the agent? Tell me, who is that
unchanging, autonomous sovereign? Then to the great benefit of
disciples, the acts of teaching, listening, and the nature of the
instruction will become perfectly clear.''

Then Bodhisattva Mahasahasrananta responded, ``O Teacher, Bhaga\-van,
the body is constructed by the mind. When matter and awareness
separate following death, the mind follows after one's karma, then it
again deludedly grasps onto the appearance of a body. Moreover, one's
body in the waking state, one's body while dreaming, and one's bodies
following this life are all constructed by the mind of
self-grasping. They are temporary transformations that have never
existed except as mere appearances to the mind. Therefore, as the mind
is the all-performing sovereign, it is of the utmost importance.  A
mindless body is nothing more than a corpse, so it has no power. When
the body and mind separate, the phenomena of feeling joy and sorrow,
reaching the state of buddhahood or wandering about in the three
realms of samsara are due to the mental consciousness becoming deluded
with respect to objects. So that is certainly the agent.  As for
speech, whatever appearances of vocalization arise, they are nothing
more than appearances to the mind. Speech has no existence other than
the phenomena of vocalization constructed by the mind's
conceptualization, so the mind is most important. When the body,
speech, and mind are separated one by one, the mind continues on; the
body turns into a corpse; and the speech vanishes altogether; so the
mind is definitely the most important.  This is the way the body,
speech, and mind are established as being non-different. In the
practice of the stage of generation, one's own body, speech, and mind
are regarded as displays of the vajra body, speech, and mind of one's
chosen deity. By so doing, they are purified, and liberation is
achieved. If they were separate, both the immutable vajra of the body
and the unimpeded vajra of the speech would be left behind when the
mind is drawn elsewhere. Then when the assembly of the three vajras
disintegrated, would the deity not perish? Therefore, they are not
different, so the many are ascertained to be of one taste.  Thus,
those three are none other than the mind; they are ascertained to be
the mind alone; and that is the best and highest understanding.''

Again the Teacher asked, ``Do you, the all-performing sovereign, have
form or do you not? If you do, what type of living being's form do you
resemble? Does that sovereign have eyes, ears, a nose, a tongue, and
mental cognition or not?  If so, where do they exist these days? Where
are they?  Moreover, do you have a round, rectangular, semicircular,
triangular, many-side, or some other kind of shape? Are you white,
yellow, red, green, or variegated in color or not? If you are, by all
means let me see this directly with my eyes or let me touch it with my
hand!  If you are certain that none of those exist, you may have
fallen to the extreme of nihilism. Therefore, now disclose to me the
reality of samsara and nirvana, joy and sorrow, appearances and the
mind, and all their substantial causes.''

Mahasahasrananta responded, ``O Teacher, Bhagavan, the self has no
form, so it is empty of form. Likewise, it has no sound, smell, taste,
touch, or mental objects, so it is empty of each of those. It is empty
of shape and color, so it is empty. The eyes, ears, nose, tongue, and
mental cognition all certainly have no existence apart from limpid,
clear consciousness itself.  Without nihilistically reducing them to
nonexistence, the indeterminate manifestations of samsara and nirvana
appear to be like an illusionist and his illusions. Thus, I have come
to the conclusion that [the mind] is simply an unceasing agent.''

The Bhagavan asked, ``O Mind Vajra, tell me, what is the source from
which you first originated? Did you originate from earth, from water
and fire, from air and space, or did you originate from the four
cardinal directions, the eight directions, from above or from below?
Investigate the place of origination and that which is originated, and
analyze them! Likewise, investigate the location in which you
subsequently are present and that which is present, and analyze them!
If this so-called mind were located in the head, when the foot is
pierced by a thorn, for instance, there would be no reason for
experiencing a sharp pain. If it were located in the feet, why would
there be discomfort even if the head and other limbs were amputated?
Suppose it were located in the body as a whole. In that case, if
unbearable regret and misery arose in the mind when an external piece
of clothing, a cup, a house, and other possessions were taken away by
others or destroyed and so on, the mind would have to be located in
those. If it were located inside, there would be no one who identifies
with things outside. If it were located outside, there would be no
clinging or grasping onto the body inside. If it is true that it is
located in the body nowadays, where will it be located when it is
separated from the body? On what will it depend?  Directly point out
the body, face, and location of the one who is present.  Investigate
the location and the environment of the one who is present, and the
size and so forth of the agent. Observe! Finally you must investigate
the act of going and the agent who goes, so observe the destination,
the path, and the point of departure on the part of the mind, the
all-performing agent, and watch how it moves. If you see the act of
going and the one who goes, show me the size of the goer and its form,
shape, and color.''

Mahasahasrananta responded, ``O Teacher, Bhagavan, I have no eyes, and
due to their absence, there is nothing that appears as form. Likewise,
I have no ears, and due to their absence, there is nothing that
appears as sound. I have no nose, and due to its absence, there is
nothing that appears as smell. I have no tongue, and due to its
absence, there is nothing that appears as taste. I have no body, and
due to its absence, there is nothing that appears as touch either.
Therefore, the five senses as well as their appearances do not exist,
so there is no one who is originated. If the one who is originated is
not established, from this time onwards the so-called mind is not
established and nonexistent.  Until now there should have been
something bearing attributes called this entity. I am unoriginated
emptiness, so the source of origination is empty. As to seeking the
place of origination, earth is something I have
constructed. Similarly, all phenomena including water, fire, air, and
space are nothing other than apparitions of self-grasping alone. So
the one who is originated is ascertained to be nonobjectifiable.  I am
non-local emptiness, so there is nowhere I am present. As for the
so-called body, sores, swelling, goiters, ulcers, and so on may arise
in the body that appears in the waking state, but they are not present
in the dream-body.  Although sores, swelling, goiters, and ulcers may
appear to have beset the body and limbs in a dream, they are not
present as phenomena in the waking state. As phenomena in the waking
state, the body may be wounded or beaten as a king's punishment, but
that does not appear on the dream-body. If that happens in a dream, it
is not present on the body in the waking state. Similarly, location,
environment, and the agent that appear to be outside or are grasped as
being inside are all nothing more than apparitions of myself alone.
Therefore, I am not present in either external or internal
phenomena. External and internal phenomena are not present in me. They
are apparitions of self-grasping, like an illusionist and his
illusions. They are not constructed intentionally as in the case of an
illusionist and his illusions. The self arises, so the appearances of
others just arise automatically, but they have no location. One may
investigate the destination and the agent, but the locus of movement
and the destination are all nonobjectifiable; so they do not apply to
the nature of me and mine. All phenomena appear while being nothing
other than the domain of the self. Moreover, that is the way the body,
speech and mind have never existed separately, and their appearances
are of the same taste. In all waking appearances, dream appearances,
and appearances after this life, the body, speech, and mind are not
different from me. Thus, it is certain that the agent of going and the
destination are not established.''

The Bhagavan replied, ``O Mind Vajra, investigate the dimensions of
your so-called mind, then establish and recognize its essential
nature. Are both external space and the internal mind the same or
different? If they are the same, the essential nature of the mind must
be space. If they are different, you would have to agree that space in
a dream, space in the daytime, and space after this life are not one
but different. If the previous space ceases and the subsequent spaces
occur one after the other, each space would be subject to
transformation, creation, and destruction. In that case, ascertain the
causes and conditions from which they arise. If space manifestly
appears in the daytime due to the sun rising in the morning, doesn't
the sun cause it to appear in a dream and after this life? Or is it
the clear light of your own mind? Don't merely give this lip-service;
rather, penetrate this with certainty.''

Mahasahasrananta responded, ``O Teacher, Bhagavan, space is
indubitably ascertained to be the essential nature of my mind. During
the daytime, earth, water, fire, air, the self, others, form, sound,
smell, taste, touch, and mental objects are displayed in the domain of
space; and the mind holds them by means of conceptualization. In dream
appearances as well, the ground of the mind appears as space, and the
entire world, its inhabitants and sense objects are all displayed as
they were before. In future lives, too, the essential nature of the
mind appears as space, and in that domain the entire world, its
inhabitants and sense objects appear in the same manner; they are held
by the mind, and one is deluded again and again.  Therefore, space,
the self, others, and all sense objects are of one taste, and they are
certainly not different. Moreover, it is the vividness of space
itself, and nothing else, that makes appearances become manifest. The
essential nature of the mind and the ground is space itself. Various
appearances occur in the realm of mental cognition --~limpid, clear,
forever-present consciousness.  The display of those appearances is
like the reflections in a mirror or the images of planets and stars in
a pool of limpid, clear water. Once limpid, clear consciousness has
entered into the central domain of pervasive, empty space, it is
directed inwards. At that time, the mind and all appearances disappear
and expand infinitely into ethically neutral, pervasive emptiness. I
have ascertained that due to self-grasping, great, pervasive
emptiness, the essential nature of the ground, arises as the mind and
mental factors. The space and vividness are none other than the
reality of the limpid, clear mind-itself, which, as a result of
conditions, falls into self and other.  Taking just the mind as the
path, a person of superior faculties directly experiences the nature
of being of thatness, which is reality-itself, realizes the
comprehensive view of samsara and nirvana, and achieves liberation in
the pristine domain of space. A person of middling faculties attains
conviction in the formless realm, and a person of inferior faculties
experiences joy in the form realm. For a person of the lowest
faculties, the path is experienced as happiness in the desire
realm. May the Teacher explain how this is so!''

He replied, ``O Mind Vajra, first of all, merge this mind with
external space, and remain in meditative equipoise for seven
days. Then fix your attention on a stone, a stick, a physical
representation of the Buddha, or a letter, and remain in meditative
equipoise for seven days. Then imagine a clear, radiant, five-colored
bindu at your heart, fix your attention on it, and remain in
meditative equipoise for seven days. As a result, for some people the
mind remains blissful, brilliant, and vacuous. That experience, devoid
of any thoughts, is like an ocean unmoved by waves, and that is called
quiescence imbued with signs. Some cannot inhibit thoughts because the
mind is so agitated, and they experience uncomfortable illness and
pain in the heart, the vital channel, and so on. Those with unstable
minds, with a wind-metabolism, or with unrefined minds may possibly
fall into unconsciousness or into a trance.  In that case, relax and
let thoughts be as they are; and continually observe them with
unforgetting mindfulness and revealing introspection. Stillness
without thinking of anything is called stillness in the domain of the
essential nature. The fluctuation and appearance of various thoughts
is called fluctuation. Not letting any thoughts remain unconscious,
but knowing them by means of mindfulness and introspection is called
awareness. With that explanation, recognize them.  Now in order to
remain for a long time in the domain of the essential nature, observe
motion, keep your body straight, maintain vigilant mindfulness, and
observe! When you say that and put it into practice, fluctuating
thoughts do not cease, nor do you get lost in them as usual; rather,
they are exposed due to mindful awareness. By striving unceasingly at
all times, both during and between meditation sessions, finally all
gross and subtle thoughts will be calmed in the empty domain of the
essential nature. You will remain and become stabilized in an
unfluctuating state, in which you will experience joy like the warmth
of a fire, clarity like the dawn, and nonconceptuality like an ocean
unmoved by waves. Craving this and believing in it, you will not be
able to bear being separated from it, and you will cling to it. If you
get caught up in joy, this will cast you into the desire realm; if you
get caught up in clarity, this will cast you into the form realm; and
if you get caught up in nonconceptuality, this will cast you to the
peak of mundane existence.  Therefore, know that while these are
indispensable signs of progress for individuals entering the path, it
is a mistake to get caught up in them forever.  That is called
ordinary quiescence of the path, and if you achieve stability there
for a long time, you will have achieved the critical feature of
stabilization in your mind-stream. However, know that among unrefined
people in this corrupt era, very few appear to achieve more than
fleeting stability. In these times, the chosen deities of some people
appear to them, and they settle their attention on the deity. To some,
visions of buddha-fields appear, and they stabilize and settle their
minds on them. Some especially experience joy, clarity, or
nonconceptuality, and they settle there. To some, images of their
spiritual mentor, rainbows, light, and bindus appear, so they settle
on those, and so forth. Know that due to the functioning of the
channels and elements of each individual, there is no uniformity in
their experiences.''

Mahasahasrananta asked, ``O Teacher, Bhagavan, please explain how
meditative experiences and realizations arise as a result of such
practice.''

He replied, ``O Mind Vajra, in all the tantras, oral transmissions,
and practical instructions of the past, awareness is nakedly
revealed. Among them, I shall not teach more than a mere fraction of
the ways signs of experience occur. Due to the unimaginable complexity
of individual metabolisms and faculties, there is a correspondingly
unimaginable array of experiences. Thus, as I am aware that there is
no uniformity, know that I shall speak only in the most general terms.
The indeterminate, unimaginable variety of avenues of experience are
inexpressible; but teachers endowed with great experience, who are
proficient in accounts of the grounds and paths, and possess
extrasensory perception are knowledgeable and lucid due to the power
of their great wisdom. On the other hand, vidyadharas up to the stages
of full maturity and mastery over their life span may be unfamiliar
with the ways in which experiences occur, and yet they know them
directly by means of extrasensory perception. Even if they do not,
they can free others from [mundane] experience by modifying the
instructions and interpreting them [for each individual].  As an
analogy, devas, risis, brahmins, acaryas, and so on, who are
practicing samadhi, may cultivate samadhi by bringing some
seed-syllables to mind. As a result, for whatever task those syllables
were used in meditation, later they can accomplish those tasks by
reciting those syllables; and on whatever illness they focus, later on
men and women can be benefited by reciting those syllables for that
illness. Likewise, vidyadharas can intuitively identify all illnesses,
or, by revealing techniques of meditation and recitation for that
purpose, they can dispel all but a few illnesses that are incurable
due to past karma. That being the case, it goes without saying that
they can elevate contemplatives' experiences onto the path.  If
foolish teachers lacking any of those qualities give instruction to
students and say that all [the above] experiences will arise in the
mind-stream of a single individual, they are deceiving both themselves
and others; and they are reduced to being maras to the lives of their
students. Why? Unpredictable outer phenomena of apparitions of gods
and demons, inner phenomena of various physical illnesses, and secret
phenomena of experiences such as mental joy and sorrow may possibly
occur. In their instructions on the mind, foolish, stupid teachers
explain the causes that give rise to such experiences, and when the
experiences occur, they fail to recognize them and instead take them
to be illnesses. They compound that by mistaking them for demons, they
think anxieties indicate death, and they insist that their students
must resort to divinations, astrology, and medical treatment. Then due
to seeing the faces of demons and malevolent beings, the students may
apply themselves to various rituals and techniques. But whatever they
do, it is only detrimental, and it does not benefit them in the
slightest, and finally they can only die. Thus, as if he had given
them a deadly poison, the teacher becomes a mara for his
students. Ponder this point carefully, and be skillful!  Now
meditation is introduced by way of terms such as insight, so there are
many explanations of the stages of the path. Here on our own path
mindfulness is presented as being like a cowherd, and thoughts as
being like cows. Their steady, vivid manifestation, without
interruption by the various appearances of hope and fear and joy and
sorrow is called enmeshed mindfulness. In general, these are some of
the signs of progress for individuals who are taking appearances and
awareness as the path: You may have the impression that all your
conceptual activity is wreaking havoc in your body, speech, and mind,
like boulders rolling down a steep mountain, crushing and destroying
everything in their path. You may feel a great pain in your heart as a
result of all your thoughts, as if you had been pierced with the tip
of a weapon. You may have the ecstatic, blissful sense that mental
stillness is a joy, but movement is a pain. You may have a vision of
all phenomena as brilliant colored particles.  You may experience
intolerable pain throughout your body from the tips of the hair on
your head to the tips of your toenails.  Or you may have the sense
that even food and drink are harmful, for you are tormented by various
afflictions included among the \oldstylenums{404} types of
identifiable, complex disorders of wind, bile, and phlegm, and so
on. You may have unidentifiable afflictions due to paranoia about
meeting other people, visiting their homes, or being in town; and you
compulsively look with hope to medical treatment, divinations, and
astrology. You may experience such unbearable agony that you think
your heart will burst. Sleep does not come at night, and even if it
does, like someone who is critically ill, you do not remain asleep.
When you wake up, your mind is filled with grief and you are
miserable, like a camel that has lost its beloved calf. Convinced that
there is still some decisive understanding or knowledge you must have,
you yearn for it just as a thirsty person longs for water. All kinds
of thoughts stemming from the mental afflictions of the five poisons
arise one after the other, and you must unbearably follow after
them. You may have any manner of experiences of speech impediments and
respiratory ailments. All thoughts, which are expressions of the mind,
and all appearances of joys and sorrows are experienced as such, so
they are called experiences; and they cannot be articulated. All
appearances of joys and sorrows are simultaneously forgotten and
vanish. You think there is some special meaning for yourself in every
external sound that is heard and form that is seen. Thinking that
there is something to be learned from the signs and omens that are
present in everything you see and feel, including the chirping of
birds, you engage in compulsive speculation. Or all external sounds
and voices, including those of humans, dogs, and birds, may seem like
thorns piercing your heart. Unbearable anger arises due to the
paranoid sense that everyone else is slandering you and speaking
against you. You may react negatively when hearing and seeing others
joking around and laughing, and with the sense that they are making
fun of you, you may verbally retaliate. Due to the experience of your
own suffering, when observing others, you compulsively long for their
happiness. With your mind filled with a constant stream of anxieties,
fear and terror arise toward weapons and even your friends; so all
phenomena give rise to every kind of hope and anxiety. In this phase,
when you get into bed at night, you may see visions of other people
who will come the next day; and when you observe visions of their
faces, forms, minds, conversations, and demons and so forth, fear,
anger, and the obsessive attachment and hatred uncontrollably arise,
and you cannot fall asleep.  Some people shed many tears due to their
reverence and devotion to their spiritual mentors, their faith and
devotion in the [Three] Jewels, their sense of renunciation and
disillusionment with the cycle of existence, and their powerful
compassion for sentient beings. Some people may find that all their
suffering vanishes, and their minds are filled with limpidity and
delight, like pristine space. Rough experiences may precede such
limpidity. You may have the sense that gods or demons are actually
carrying away your head, limbs, and vital organs, leaving behind only
a trail of vapor; or you may merely have the sensation of this
happening, or it may occur in a dream. Afterwards, all your mental
suffering vanishes, and you experience a sense of ecstasy like the sky
free of clouds. In the midst of this, four mindfulnesses and various
kinds of pleasant and harsh sensations may occur. Spiritual friends
who teach this path properly must know and realize that those are not
the same for everyone, so bear this in mind!  For one with a fire
metabolism, a sense of joy is prominent; for one with an earth
metabolism, a sense of obscuration is prominent; for one with a water
metabolism, a sense of clarity is prominent; for one with a wind
metabolism, harsh sensations are prominent; and for one with a space
metabolism, a sense of emptiness is prominent. After all pleasant and
harsh sensations disappeared into the absolute nature, there is no
need to modify thoughts. Rather, by letting them be as they are,
everything that arises loses its capacity to help or harm, and you
dwell in that state. You may also have an exceptional sense of joy,
clarity, and nonconceptuality, visions of gods and demons, and a small
degree of extrasensory perception. The channels and elements function
differently from one person to the next, so those with dominant earth
and wind elements commonly do not experience extrasensory perception
or visionary experiences. Extrasensory perception and visions happen
chiefly to fire-element and water-element people.

Now here are the different levels classified by name: superior vision
with single-pointed mindfulness, in which movement and mindfulness are
unified, is called insight. At this time if the sense of stillness
predominates, that is called the union of quiescence and insight. In
what way is that vision superior? Previously, even if you watched with
great diligence, you were obscured by subconscious movement and by
laxity and dullness; so thoughts were hard to see. But now, even
without exerting yourself very much, all thoughts that arise become
conscious, and you see them exceedingly well. As for the experiential
visions at this stage, to some contemplatives, wherever they look,
everything that arises appears as divine embodiments and as vibrant
bindus. To some, a variety of different seed-syllables, various
lights, and all manner of forms appear. To some appear buddha-fields,
unknown regions, melodies, songs, and speech by various unknown
beings, and indeterminate varieties of many viras and dakinis dancing
and displaying various expressions. To some, all sights, sounds,
smells, tastes, and tactile sensations appear as all manner of signs
and omens. Some have the sense of observing many entities with and
without form by means of extrasensory perception.

After becoming thoroughly familiar with that, any kind of sense of joy
or sorrow may trigger a unification of mindfulness and
conceptualization. Then, like the knots in a snake unraveling,
everything that appears dissolves into the external environment; then
everything appears to vanish by itself, resulting in a natural
release. Appearances and awareness become simultaneous, so things seem
to be released simply upon being known. Thus, arising and release are
simultaneous. As soon as things merely arise from their own space,
they are released back into their own space, like lightning that
flashes from the sky and vanishes back into space. Since this appears
by observing within, this is called space-release. All these are in
fact the unification and concentration of mindfulness and appearances.
After all pleasant and unpleasant experiential visions have dissolved
into the absolute nature, consciousness rests in its own state of
immaculate limpidity.  Whatever thoughts and memories arise, do not
cling to these experiences, modify them, or judge them, but let them
manifest as they roam about in their own dynamic state. The effort of
vivid, steady apprehension, as in the case of the conceptualization of
enmeshed mindfulness, dissolves by itself.  Consequently, the
unsatiated mind compulsively strives after mental objects.  With a
sense of unfulfillment and deficiency, at times you compulsively
engage in much cogitation entailing tight concentration and so on. In
this phase, consciousness will come to rest in its own state,
mindfulness will manifest, and with little clinging to experiences, it
will settle into its own natural state free of modification. Thus, you
come to naturally settled mindfulness.  The experiential sense of that
is soothing and gentle, with clear, limpid consciousness that is
neither benefited nor harmed by thoughts; and you experience a
remarkable sense of stillness without needing to modify, reject, or
embrace anything. Thus, if you are not counseled by a good spiritual
friend at this time, you may think, `Now there has arisen in my
mind-stream an extraordinary, unparalleled view and meditative state
that is difficult to fathom and to be shared with no one else.' As
time passes while you place your trust and conviction in that, you may
discuss it with no one else and delude yourself. When you come to this
situation, even if you speak of it to another spiritual friend, unless
that person is skilled in speaking and knows how to listen critically,
your knowledge of the path will be terribly mistaken.  Therefore, if
you get stuck there and your life passes by, that error will tie you
down so that you will not transcend the realm of mundane existence. So
be aware of this!

In particular, a sense of clarity may result in visions of gods and
demons, and in your thoughts you may appear to be suddenly assaulted
by demons. At times that may be true. But by regarding that as
extrasensory perception and repeatedly dwelling on gods and demons, in
the end demons will seem to proliferate. Then by conceptually
fabricating gods and demons, and proclaiming your extrasensory
perception to others, finally you will be practicing demon meditation,
and your mind-stream will be possessed by demons. Your vows and
samayas will then deteriorate, and you will stray far from Dharma, get
lost in mundane activities of this life, and delude yourself in magic
rituals. Without even an iota of contentment, you will follow after
food and wealth, and your mind will be bound up with clinging,
attachment, and craving. While in that state, when you meet with
death, you will be reborn as a malevolent demon. By accumulating the
causes of experiencing the environment and suffering of a sky-roving
preta, your view and meditation go awry, and you become deluded in the
endless cycle of existence.

When people of medium and inferior faculties have entered this path,
indications of the path will inevitably occur, but if they cling to
anything, they will be trapped again by that clinging. Knowing that
such experiences are highly misleading and unreliable, leave your
awareness in its own state, with no clinging, hopes, fears, rejection,
or affirmation. By so doing, those experiences will be naturally
released in their own nature, like mist disappearing into the sky. So
be aware of that!  O Mind Vajra, specific types of good and bad
experiences are unpredictable. All techniques from the time of
quiescence until conscious awareness becomes manifest are solely means
of bringing about experiences, so anything may happen. Therefore, know
that ascertaining everything as experiences is a crucial point and the
quintessence of practical advice. Then realize that and bear it in
mind!''

Then Mahasahasrananta asked, ``O Bhagavan, thus if all pleasant and
rough experiences are distant from the path to omniscience and are of
no benefit, why should we practice meditation? May the Teacher
explain!''

The Bhagavan replied, ``O Mind Vajra, individuals with unrefined,
dysfunctional minds agitated by conceptualization enter this path; and
by undermining the potency of compulsive ideation, their minds become
increasingly steady, and they achieve unwavering stability. On the
other hand, even if people identify conscious awareness, but do not
practice, they will succumb to the faults of spiritual sloth and
distraction. Even if they do practice, due to forgetfulness, they will
get lost in endless delusion.  The mind --~which is like a cripple~--
and vital energy --~which is like a blind, wild stallion~-- are
disciplined by fastening them with the rope of experiences and
grasping. Once people of dull faculties have discerned the mind, they
control it with the cords of mindfulness and introspection. By so
doing, due to experience and habituation, they have the sense that all
subtle and gross thoughts have vanished. Finally, there arises a state
of unstructured consciousness devoid of anything on which to
meditate. Then when awareness reaches the state of great
non-meditation, the spiritual mentor points that out, so they do not
go astray.  For those purposes, first one undergoes great struggles in
seeking the path, one takes roving thoughts as the path, and finally
when consciousness comes upon itself, that is identified as the
path. Until the unstructured awareness, or consciousness, of the path
manifests and comes upon itself, due to the arousal of one's afflicted
mind, one must gradually proceed through rough experiences like those
discussed previously.''

Bodhisattva Mahasahasrananta asked, ``O Bhagavan, are thoughts to be
cleaned out or not? If they are, must consciousness arise again after
the mind has been purified? May the Teacher explain!''

The Teacher replied, ``O Mind Vajra, the ties of mindfulness and
grasping are dissolved by the power of meditative experiences until
finally the ordinary mind of an ordinary sentient being, as it were,
disappears. Consequently, compulsive ideation becomes dormant, and
roving thoughts vanish into the absolute nature. The vacuous
total-ground descends to a state in which the self, others, and
objects disappear. One gazes inwards upon clear emptiness with a sense
of clinging, and the appearances of self, others, and objects
vanish. That is the total-ground consciousness. Some teachers explain
the descended total-ground as the one taste and as freedom from
conceptual elaboration, and others say it is ethically
neutral. However it is described, in fact one has come upon the
essential nature.  Again, a person filled with zeal ascertains that
this is not the real path,\footnote{Even though one has come upon the
  essential nature of awareness, if one is complacent with having
  ascertained that once, that is not the real path, just as taking
  refuge alone is not sufficient for achieving Buddhahood. Only by
  continuing in one's meditation practice does all clinging and
  grasping cease.} and as a result of meditating, all such emptiness
and clear emptiness that are imbued with experiential clinging vanish
into the absolute nature as if one were waking up. Thereafter, outer
appearances are not impeded, and the threads of inner mindfulness and
grasping are cut. Then one is not bound by the ties of good
meditation, and one does not fall back to an ordinary state due to
pernicious ignorance. There is forever-present, limpid, luminous
consciousness that transcends the conventions of view, meditation,
and, conduct. There is no determination of self and object, such that
one can say `this is consciousness' and `this is the object of
consciousness.' There is freedom from clinging in a primordial,
self-arisen state in which the mind has experiences. When you come
upon a spaciousness in which there is no cogitation and no attentional
referent, all phenomena become manifest, for the power of awareness is
unimpeded. Thoughts merge with their objects, they disappear as they
become nondual with those objects, and they become
disintegrated. Since not even one has an objective referent, they are
not thoughts of sentient beings; rather, the mind has been transformed
into wisdom; the power of awareness is transformed, and stability is
achieved there. Know that this is like water cleared of sediment.''

\section{The Questions of Prasannatindra}

Then Bodhisattva Prasannatindra rose from his seat and addressed the
Bhagavan, ``O Teacher, Bhagavan, may the Omnipresent Lord, the
Immutable Sovereign, listen and attend to me. Is the state of
primordial, self-arisen liberation achieved solely by cultivating
clear awareness that is inconceivable and ineffable, or is it not? If
it is, how is it achieved? If it is not, what is the point of
cultivating this? What kind of excellent qualities arise? Please
explain this for the sake of disciples.''

The Teacher replied, ``O Prasannatindra, you listen and bear this in
mind, and I shall truly explain this to you. If the forever-present
primordial wisdom of the dharmakaya is not realized even though you
have practiced achieving stability in this profound path in the very
state of conscious awareness free of conceptual elaboration, as soon
as you pass away from this life, there are forces to propel you to the
form realm and the formless realm;\footnote{The Tibetan reads mu
  bzhi'i khams, literally translated as ``realm of the four
  alternatives'', instead of the more common gzugs med khams. The
  ``four alternatives'' are the four meditative absorptions of the
  formless realm, namely, infinite consciousness, infinite space,
  nothingness, and neither-discernment-nor-nondiscernment.} but with
this alone it is impossible to achieve omniscient buddhahood. Once you
have initially identified this path, if the forever-present,
primordial wisdom of the dharmakaya is identified due to the power of
intense meditation, that is the wisdom of the path and the creative
power of primordial wisdom.  These are the resultant excellent
qualities. There is no other space apart from the space inside a pot,
and there is no water superior to the water that fills a
cup. Likewise, there is no path other than this path of manifest,
conscious awareness. Even if you wander down into the impure cycle of
existence, that is constructed by the stream of consciousness; and
even if, with the virtuous karma of lofty merit, you practice creating
deities, meditation, and recitation, that is accomplished with the
stream of consciousness; and even if you practice transforming the
channels, bindus, and vital energies into displays of the three
vajras, it is the stream of consciousness that liberates. Moreover,
the manifestation of this alone is the primordially pure ground, which
is self-arisen, limpid, clear, nondual, primordial wisdom.  In
general, whatever spiritual vehicle you enter, there is no entrance
apart from the stream of forever-present primordial
wisdom. Furthermore, when ordinary, deluded sentient beings focus on
virtue as they chant recitations many times over, that is taught for
the sake of the stream of ground-consciousness. Therefore, the stream
of consciousness accumulates all karma, so this manifest consciousness
itself is unrivaled by any defiled virtue. Moreover, the difference
between practicing a mere technique pertaining to the stream of
consciousness on the one hand and manifesting consciousness on the
other is like that between the sky and the earth. Thus, all
extraordinary, excellent qualities are perfected in this manifestation
of consciousness.  O son of good breeding, forever-present, primordial
wisdom is the inherently clear light of the minds of sentient
beings. It has no referent with respect to all appearances and minds,
and the manifestation of this is the outer luster of wisdom. The
nature of that manifestation is the inner luster of wisdom. That is
very different, like the dawn appearing in the sky. If energetic
people of superior faculties apply themselves uninterruptedly and
single-pointedly to practice, with no essential diversions, finally
the creative power of discerning primordial wisdom will be ignited. As
a result, the excellent qualities of the view and meditation of the
clear light Great Perfection, which is reality-itself, the very nature
of suchness, will truly manifest; and they will become spiritually
Awakened as Samantabhadra, the original, primordial ground. Even
individuals who are not of that sort may identify this crucial point
of unstructured, self-arisen consciousness, which manifests without
meditation, and they may achieve a little stability in that. All other
physical and verbal virtues accumulated throughout a galaxy would not
come close to the merit of even one part in a hundred, a thousand, ten
thousand, or a hundred thousand of that. Those people would certainly
achieve long-lasting stability in the peak of mundane existence.''

Again Bodhisattva Prasannatindra asked, ``O Teacher, Bhagavan, may the
omnipresent Lord, the Immutable Sovereign listen and attend to me. No
matter how much one meditates in that way by taking the mind and
consciousness as the path, if that does not result in the fruitional
state of liberation or in buddhahood, please show us a method for
directly identifying for ourselves the primordially pure Great
Perfection, sovereign awareness free of extremes, without having to
resort to such a long and difficult path that yields various joys and
sorrows but no accomplishment of the fruition. Reveal to us the stages
of the path free of hardships, and give us profound, practical
instructions to prevent us from falling into error.''

He replied, ``O son of good breeding, the great, universal ground of
all spiritual vehicles is profound emptiness. I shall explain the way
to determine the reality of profound emptiness, so listen well! The
basis of delusion of all sentient beings of the three realms is
ignorance of oneself alone. Examine the basis and root of its origin,
location, and departure. To investigate the basis and root of the
initial origin of the `I': there is a stream of consciousness that
grasps onto the fundamental, pervasive, surrounding space as the
self. All appearances and mental states are nonexistent and not
established except as mere appearances, so their source of origin is
empty.

To investigate its location where it dwells in the interim: the head
is called the head, and it is not given the name `I'; likewise, hair
is hair and not `I'; the eyes are eyes and not `I'; the ears are ears
and not `I'; the nose is the nose and not `I'; the tongue is the
tongue and not `I'; the teeth are teeth and not `I'; the
shoulder-blades are shoulder-blades and not `I'; the upper arms are
upper arms and not `I'; the lower arms are lower arms and not `I'; the
palms, the back of the hands, and the fingers are not `I'; the spine
is not `I'; the ribs are not `I'; the lungs and heart are not `I'; the
liver and its lining are not `I'; the small intestine, spleen, and
kidneys are not `I'. The thighs, hips, calves, ankles, and all the
finger and toe-joints each have their own names, and they are not `I'.
The skin, fat, flesh, blood, lymph, ligaments, tendons, and body hair
all have their own names, and they are not established as `I'. If the
`I' were to be located in the lower part of the body, there would be
no pain if the head and the upper limbs were amputated, so it is not
present there. If it were located in the upper body, there would be no
pain if lower portions of the body such as the legs were harmed. If it
were located inside, there would be no reason why searing pain would
be experienced due to the outer body hair and skin being scraped
off. Consider whether it is located in the body. When all your
clothes, jewelry, food, wealth, and possessions are taken away and
used by someone else, misery and intolerable attachment and hostility
arise, so it is not located there. Consider whether it is located in
external objects. The entire world and its inhabitants would be
apprehended as being mine, but in fact all things have their own
names, and they are not `I'. Even if all appearances of the world and
its inhabitants apart from the `I' seem to exist individually, among
all dream phenomena, phenomena of this life, and phenomena following
this life, the self and all appearances seem to be like the body and
its shadow, like liquid and moisture, and like fire and heat. Thus,
the `I' dominates the entire world and its inhabitants, but the `I' is
not located anywhere.  Finally, to investigate and analyze its
destination: the entire phenomenal world is the basis and essential
nature of the great phantasm of `I', so its destination is naturally
empty. All the three realms arise from apparitions of grasping onto
the `I', so it has nowhere else to go. The being who is an agent does
not originate and it is located nowhere, so it follows that it
disappears.''

Again Prasannatindra asked, ``Thus, if it is certain that its origin, location,
and destination do not exist and are not established, how do you account for
qthe continuity from one appearance to the next? May the Teacher
explain!''

He replied, ``O son of good breeding, after the consciousness that
grasps onto the `I' has manifested, `I' and 'mind' emerge from their
own space, and they disappear back into their own space. They
alternately emerge and withdraw into the vastness of the ethically
neutral ground of emptiness. Thus, dream phenomena, phenomena in the
waking state, and all the phenomena of the three realms nonexistently
arise as mere appearances. Therefore, know that the location of
motion, the destination of one who is moving, and the agent are not
established.''

Again Bodhisattva Prasannatindra asked, ``O Teacher, Bhagavan, when
grasping onto the `I' vanishes into the absolute nature, is its
continuum not severed? May the Teacher explain!''

He replied, ``O son of good breeding, even when the appearances and
mental states of grasping onto the `I' vanish into the absolute
nature, the ethically neutral ground whose excellent qualities are not
manifest acts as the cause of self-grasping. Thus, just that unceasing
continuum of the causal ignorance of self-grasping is called the
grasping onto the identity of a person.''

Prasannatindra asked, ``Teacher, Bhagavan, in what way are external
objects empty? May the Teacher explain!''

He replied, ``As for the emergence of grasping onto the identity of
apprehended phenomena other than `I', let us investigate the way in
which all these names, things, and signs are not established. First of
all, let us determine the emptiness of the names of the body by
investigating the bases of designation of names. To examine that which
is called the head: hair is hair and not the head; the eyes are eyes
and not the head; the ears are ears and not the head; the nose is the
nose and not the head; and the tongue is the tongue and not the
head. Likewise, the skin, flesh, bones, blood, lymph, ligaments and so
on all appear to have their own names, so they are not established as
the head.''

Prasannatindra asked, ``Teacher, Bhagavan, if you reduce it to its
components like that, it is not established, but isn't their
configuration called the head?''

He replied, ``Son of good breeding, observe that in general there are
many cases in which the assembly of those components is not designated
as a head. If one person's head were reduced to particles, which were
then assembled and shown to others, they would not call it a
head. Even if those particles were moistened and formed into a sphere,
that would not be designated as a head.  If your head that appears at
the time of a dream, your head that appears during the waking state,
your head that appears in the past, and your head that appears in the
future were all identical, whatever sores, swellings, goiters, moles,
and warts you had would have to appear on all those occasions; but
they do not. If each of those heads were different, either all the
prior heads would have to be cast off, or else this would indicate the
oversight that they were not established from the beginning. If you
say it is called the head because it is seen to be on top, you should
analyze the upper and lower regions of space.  Thus, by investigating
how the front, back, upper, and lower regions of space exist, you will
determine that it is not established in any of them.  Likewise, upon
what is the eye designated? All fluid spheres are not known by the
name of eye. The skin, blood, fat, channels and sinews are not granted
the label eye. As in the previous case, the eye does not exist as
their assembly either. If you think that a fluid sphere that sees
forms is called the eye, observe whether that which sees forms at all
times in the past, future, and present, in dream appearances and
waking appearances, is this fluid sphere of the
present. Self-appearances are due to primordially present
consciousness rather than this fluid sphere of the present. Even if
you took in hand a hundred million eyeballs facing in one direction,
they would not see form.  Likewise, as for the ears, if the flesh,
skin, channels, sinews, blood, lymph, and cavities each has its own
label and not the appellation of 'ear,' what is called the ear? If you
say something is called the ear because it hears sounds, investigate
whether that which hears sounds at all times during and after this
life, while dreaming and while awake, is the ear. By so doing, you
will find that it is the consciousness of the mind that hears and not
the form of the present ear. Even if you had in hand countless ears
that were listening, they would not hear sounds. This being the case,
the ear is primordially not established.

Similarly, by investigating and analyzing the label and real signs of
the nose, you find that the flesh, bones, blood, lymph, channels,
sinews and cavities all have their own different labels, so they are
not established as the nose, nor is it established in their assembly.
If you think that that which smells odors is called the nose and that
odors are sensed from this orifice, consider that this orifice is not
needed in the dream-state nor in other lifetimes. Consciousness in the
intermediate state senses odors as well. Therefore, since mental
consciousness has no nose, the nose certainly has no objective
existence.  Likewise, the tongue is not established in any of the
individual components of the flesh, blood, skin, channels and sinews;
nor is the label of tongue established in their assembly. If you
assert that this which experiences tastes is the tongue, investigate
whether or not it is this very tongue that experiences tastes in the
dream-state, the intermediate state, and in other lifetimes; and then
you will know.

By investigating the so-called body in terms of the skin, fat, flesh,
blood, marrow, bones, and all the channels and sinews, you will find
that the body is not established. If they were all reduced to small
and minute particles and then massed together, the label of body would
not apply. Even if they were moistened and formed into a lump, that
would not be a body. If you say that that which experiences tactile
sensations is designated as the body, examine: who is the experiencer
of tactile sensations while in a dream and the intermediate state? By
so doing, that is determined to be mental consciousness
itself. Therefore, since the label of body is not applied to the mind,
the body does not exist.

Moreover, upon investigating the locus of the so-called arm, the
shoulder is not the arm, nor is the upper arm, the forearm, or the
palm and fingers the arm; so I say, `Identify what the arm is and tell
me.' You may say that this which performs the functions of the arm is
called the arm. But then, examine whether this is that which appears
as an arm and performs the functions of an arm in a dream, and whether
it is everything that appears as such in the intermediate state. If
you do so, you will find that it is not. Rather, you will determine
that they are merely appearances to the mind; so the arm is not
established except as something imputed upon the mind.

Moreover, upon examining the shoulder, the flesh is not the shoulder,
nor are the bones, channels, and sinews. It is not established in any
of those individual components, and it is not the assembly of the
particles to which they can be reduced, even if you were to moisten
them and form them into a lump. Likewise, by carefully examining all
the joints, it is ascertained that the basis of designation of that
label has no objective existence.  Furthermore, upon what do you
designate the name for the appearance of a human being over there? The
head is not a human. The five sense faculties are not a human. The
label of human is not established in flesh, blood, bones, marrow,
channels, sinews, major and minor limbs, or consciousness. Likewise,
what is the basis of designation of house? Earth is not a house; as
for stone, the name for house does not apply to it, only stone
itself. The label of house does not apply to the pillars, rafters,
beams, or foundation; and even if they were piled together, the label
of house would be unwarranted. For example, with respect to a cup, its
exterior is not a cup, nor is its interior, its mouth or its base, and
wood is not a cup. Neither its individual components nor their
assembly exist objectively as its basis of designation. Also in the
case of a so-called mountain, earth is not the mountain, nor are
stones, grass, or trees; and their assembly is also not a mountain. So
the name mountain is empty .  To examine the basis of designation of a
single stick, its tip is just a tip and not a stick. Its base is
nothing other than its base; the wood is nothing other than wood; its
burnt ashes are nothing other than ashes; and its ground-up particles
are just particles and not a stick. So even that label can only vanish
without any objective existence.  Know that earth, water, fire, and
air also do not exist in the realm of gross particles, minute
particles, or partless minute particles. As for all manner of labels,
an illusion is nonexistent and is nothing more than the mere label
illusion; a mirage is nonexistent and is nothing more than a mere
label; a dream is nonexistent and is nothing more than a mere label; a
reflection is nonexistent and is nothing more than a mere label; a
city of gandharvas is nonexistent and is nothing more than a mere
label; an echo has no objective existence apart from its mere label of
echo; the moon in water is nothing more than the mere words the moon
in water; a water bubble has no objective existence apart from the
mere words water bubble; an optical illusion has no objective
existence apart from its mere label; and a magical emanation has no
existence apart from the mere utterance of its name. Like the
utterances of the sounds of those names, all bases of designation of
uttered names and words of all manner of appearing phenomena are
nonexistent; and they are emptiness, which is not
established. Recognize that emptiness has no objective existence, and
emptiness wholly terminates in the expanse of space. That is practical
advice.''

Again Bodhisattva Prasannatindra asked, ``O Teacher, Bhagavan, foolish
disciples who are bound by clinging onto true existence cannot realize
the mode of existence of fundamental reality-itself simply by way of
the nonexistence of the basis of designation of a label. Thus, may the
Teacher reveal a way to determine its nonexistence by way of gross and
subtle inquiry.''

He replied, ``O Prasannatindra, it is like this: if a tree trunk
appears, is the existence of this trunk permanent, or is it utterly
nonexistent? This is the investigation and analysis. If it were
permanent, it could not be cut or destroyed, and it would have to be
truly existent, impermeable, immutable, impenetrable, and imperishable
in every way. But a gash occurs when it is chopped with an ax, so it
can be cut, and when it is chopped many times, it is felled and
destroyed. Since one becomes many, it is deceptive and not truly
existent. Since it can be stained with white and black dyes and
powders, it is not impermeable but permeable. Since it is subject to
change due to the season:; and circumstances, it is not immutable. It
can be penetrated and destroyed anywhere, so it is penetrable,
perishable. This wood can be demolished in any number of ways, so it
is perishable. As it does not have even one of the qualities of a
vajra, it is ascertained to be nonexistent.''

Prasannatindra asked, ``O Teacher, Bhagavan, please explain what is a
vajra replete with all the seven qualities of a vajra.''

He replied, ``O Prasannatindra, referring to the existence of a
conventional, material vajra is like referring to the son of a barren
woman. Material vajras are shown to be made of bone; stone vajras can
be destroyed by incinerating them; and iron vajras are melted in
fire. They are not truly existent; rather, those conventional vajras
are destructible. This space-vajra, which appears everywhere, cannot
be cut with weapons or anything else. It cannot be destroyed by
objects or circumstances. Devoid of faults or contamination, it is the
great basis of the expansion of the cosmos, so it is real. It cannot
be contaminated by faults or virtues, so it is impermeable. It is free
of change throughout the three times, so it is immutable. Since
everything is pierced in emptiness, it is totally impenetrable. It
cannot be modified or changed by anything, so it is utterly
imperishable.

This is the space-vajra that appears everywhere. For those who grasp
onto substantiality, this is a conventional vajra, and for those who
comprehend its nature as unadulterated liberation, this is the
ultimate, indestructible vajra.  If there were some other object
replete with all the seven qualities of a vajra, it would be
permanent; but if not, then it is certain that everything is
emptiness, which is not established.  Thus, substances that appear as
things such as a tree trunk, earth, stone, buildings, and household
goods may be pounded, broken, and ground up. By grinding them down to
particles, they are reduced to powder. By pulverizing those particles
to one-seventh their size, they are reduced to minute particles; and
by disintegrating them to one-seventh their size, they are reduced so
that they become partless. They are obliterated and vanish into the
nature of space.

Moreover, the ashes of any substance that has been burned in fire
naturally disappear into space, and something that appears to be the
form of a living being vanishes altogether once it has been killed and
burned up. By examining and analyzing all such appearing phenomena in
that manner, you find that they all vanish completely, and not a
single thing is established in true existence. Intensive inquiry into
that topic is essential, so know this!''

Prasannatindra asked, ``O Teacher, Bhagavan, if they are not
established and are unreal in that manner, whence do all phenomena
arise and appear? May the Teacher explain!''

He replied, ``O Prasannatindra, with self-grasping acting as the cause
and conceptualization acting as the contributing condition, they exist
as mere appearances. The initial consciousness moves to the object,
suddenly arousing an appearance. Due to the thought to eliminate it
and the appearance of thinking it will be destroyed, it shifts or
vanishes altogether. All phenomena are nothing more than mere
appearances from dependently related events. There is certainly
nothing whatever that is truly existent from its own side.  For
example, due to the simultaneous assembly of someone else's eyes as
the cause, with limpid, clear space serving as the basis, and with the
substances and mantras used for an illusion and the emanating mind as
the contributing conditions, the dependently related event of an
illusory emanation appears even though it is nonexistent. Due to the
assembled, dependently related events of limpid, clear space as the
cause, and warmth and moisture as the contributing conditions, a
mirage appears that is not established. From the dependent
relationship of the limpid, clear, total-ground consciousness as the
cause and self-grasping as the contributing condition, dream
appearances arise that are nonexistent; and one is deluded by grasping
onto their reality and clinging to their true existence as if they
were appearances in the waking state. Due to the dependent
relationship of the simultaneous proximity of a limpid, clear mirror
as the cause and someone's face as the contributing condition, a
reflection appears that is nonexistent. Due to the dependent
relationship of the samadhi of the cultivation of meditative
stabilization as the cause and due to the simultaneous proximity of a
vessel and moisture as the contributing condition, a city of
gandharvas appears as an object. Due to the dependent relationship of
a solid, high object such as a boulder and audial consciousness as the
cause and making a noise such as shouting as the contributing
condition, an echo occurs. Due to the dependent relationship of
limpid, clear water as the cause and the simultaneous arising of
planets and stars in the sky as the contributing condition,
reflections appear. Due to the dependent relationship of water itself
as the cause and stirring or agitation as the simultaneous,
contributing condition, bubbles emerge. Due to the dependent
relationship of the eyes as the cause and simultaneous pressure
applied to the eye-channels as the contributing condition, an optical
illusion takes place. Due to the dependent relationship of mastery in
emanating as the cause and entrance into the samadhi of producing
emanations as the contributing condition, nonexistent emanations
appear.

Thus, for those ten analogies there is dependence due to reliance upon
causes, there is a relationship due to the nonduality of the causes
and the contributing conditions, and there is origination due to the
emergence of nonexistent appearances. In a similar fashion, there
arises a continuum of consciousness of self-grasping onto the `I' with
respect to the unobstructed, nonobjective displays of the groundless,
rootless domain of space, which is the fundamental, absolute nature of
the pervasive realm of space. Due to that stream of consciousness, the
ground is fragmented: by retracting the self, the fundamental,
absolute nature is externalized. From the limpid, clear, mirror-like
ground, in which anything can arise, the appearances of the three
realms spring forth. As analogies, due to the emergence of foam that
is nondual with the ocean, the ocean is set apart; and due to the
appearance of rainbows in the sky, which are none other than the sky,
the sky appears as something else.  There is dependence due to
reliance upon the `I', there is a relationship due to the nonduality of
self and other, and there is origination due to events that have no
objective existence. Thus, by investigating the mode of existence of
all kinds of appearing phenomena, recognize the crucial point of
ascertaining them as displays of the empty space of reality-itself.

Moreover, when you fall asleep, the inanimate world of appearing
objects during the waking state, the sentient beings who inhabit the
world, and all the appearing objects of the five senses dissolve into
the vacuous total-ground, which is of the nature of space; and then
they emerge from that domain. Once again self-grasping consciousness
is aroused due to the apparitions of the movements of karmic
energies. Consequently, due to self-appearances, everything inside and
outside, including the inanimate and animate world and sensory
objects, emerges as dream appearances, like before, in the
fundamental, absolute nature. Joy, sorrow, and indifference are
closely held and clung to as being truly existent. This is delusion,
so recognize it!''

\section{The Questions of Samantabhasavyuhendra} % vajra_heart_3.txt

\end{document}
