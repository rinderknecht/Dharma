%%-*-latex-*-

\documentclass[11pt,a4paper]{article}

% Language and fonts
%
\usepackage[utf8]{inputenc}
\usepackage[T1]{fontenc}
\usepackage[french]{babel}
\usepackage{hyphenat}
\usepackage{ctib}
%\usepackage{color}

\def\myhung{\ooalign{%
  {\raise0.7\fontdimen5\font\hbox{\hskip0.15em\tibNyizlanaada}}%
  \cr\V{h}{\raise0.2em\hbox{\hskip0.25em\scriptsize 'u\notsheg}}\notsheg}%
}

% Miscellanea
%
\usepackage{xspace}

\newcommand\yangti{Yangti Nagpo\xspace}

% Main document
%
\title{\yangti}
\author{Ketsün Sangpo Rinpoche}
\date{24--27 Juin 2008}

\begin{document}

\maketitle

\paragraph{Préliminaire.}

Tout d'abord, il est important de méditer sur les mérites nécessaires
qui, accumulés dans des vies antérieures, nous ont amené à ce moment
où l'on reçoit ces précieuses instructions. Il faut aussi examiner sa
motivation et s'assurer qu'elle est la libération de tous les migrants
des souffrances de l'existence cyclique.

\paragraph{Le \yangti et les Neufs Véhicules selon l'École des
  Anciens.}

L'atiyoga est le véhicule supérieur et est divisé en quatre
catégories, externe, interne, secrète et ultime, et est présenté selon
trois sections (ou cycles): espace, esprit et préceptes. Le Yangti est
la vue ultime de la Grande Perfection. Le mot tibétain «~nagpo~»
signifie l'obscurité. Le Yangti Nagpo consiste en deux niveaux:
\emph{thögal} et \emph{trekchö}. Le premier consiste en trois voies,
classées selon une couleur: le blanc, le jaune et le noir. La voie
blanche est fondée sur l'observation de la lumière solaire dans le
ciel; la voie jaune est basée sur la contemplation de la lumière
lunaire et la voie noire consiste à prendre l'obscurité comme
support. La voie du Yangti Nagpo n'est donc pas restreinte aux
retraites dans l'obscurité et l'interprétation du mot «~nagpo~» est
analogique: de même que le noir recouvre toutes les couleurs, mais
n’est recouvert par aucune, de même, la nature de \emph{rigpa}
enveloppe tous les phénomènes mais n’est voilé par aucun d’entre eux.

\paragraph{Préliminaires Ordinaires et Extraordinaires.}

Le pratiquant souhaitant s'engager dans la voie du \yangti doit, en
principe, avoir terminé les accumulations méritoires du \emph{ngöndro}
du Yangti.

\paragraph{Hommages.} Le texte débute par un hommage à l'Éveillé
primordial Samantabhadra, qui symbolise le \emph{dharmak\=aya}, car
cette pratique révèle la réalité nue, c'est\hyp{}à\hyp{}dire le
c{\oe}ur des \emph{\d{d}\=akin\=\i}. Le \emph{sa\d{m}bhogak\=aya} est
ici représenté par Vajrasattva et le \emph{nirm\=a\d{n}ak\=aya} par
Guru Rinpoche (ce qui place le texte dans le cadre de la Grande
Perfection).

\paragraph{Les Dix Particularités du \yangti.}

\begin{enumerate}

  \item Les Terres pures sont vues directement, au lieu d'être
    visualisées (comme dans les \emph{tantra} par example);

  \item la réalisation obtenue est identique à celle définie par le
    M\=adhyamika et, de plus, est plus facile à obtenir;

  \item la vue est intégrée à l'esprit mondain (au lieu d'être
    réalisée pendant les périodes de contemplation uniquement);

  \item le corps du pratiquant est perçu par celui\hyp{}ci comme la
    source du déploiement des \emph{ma\d{n}\d{d}ala} des Cinq Familles
    d'Éveillés;

  \item les déités ne sont pas visualisées mais jaillissent
    spontanément de la luminosité et des sphèrules quinticolores
    (\emph{bindu});

  \item les cinq sagesses sont vues par le biais des yeux;

  \item il n'est pas besoin d'attendre l'état intermédiaire
    (tib. \emph{bardo}) entre la mort et la vie pour voir le
    déploiement des cinq sagesses;

  \item il est possible d'atteindre le plein Éveil en sept ou
    vingt\hyp{}et\hyp{}un jours;

  \item la luminosité unit le jour et la nuit en une même continuité;

  \item le corps est perçu comme luminosité des sagesses et peut
    entièrement réintégrer cette luminosité primordiale (hormi les
    phanères).

\end{enumerate}

\paragraph{Le chemin.}

Il faut d'abord achever les pratiques préliminaires ordinaires et
extraordinaires, puis recevoir quatre initiations et enfin l'on peut
pratiquer selon les instructions du maître.

\paragraph{Préparation.}

La première étape du \yangti proprement dit consiste en des pratiques
propédeutiques ordinaires et extraordinaires. Les premières ne sont
autres que le \emph{ngöndro}. Les secondes ont pour objectif de
purifier le corps, la parole et l'esprit. Il est très important de
conserver ces instructions secrètes.

\begin{itemize}

  \item \textbf{Corps.} Après avoir élu un lieu isolé, se tenir debout
  sur la pointe des pieds dirigés vers les côtés, les talons se
  touchant au centre et les paumes jointes au\hyp{}dessus de la tête
  sans la toucher. La forme du corps symbolise un \emph{vajra} à trois
  pointes et doit être alors visualisé comme tel, irradiant une
  lumière blanche intense.\footnote{Ou entourée de flammes rouges?}
  Des myriades de petits \emph{vajra} sont alors émis par le corps
  ainsi visualisé, dans toutes les directions et ils détruisent tous
  les obstacles matériels sur leur passage. Enfin, lorsque le monde
  extérieur est ainsi purifié, les \emph{vajra} reviennent au corps et
  s'y dissolvent. On itére ensuite cette session en utilisant les
  quatre autres couleurs des Cinq Familles, sans ordre particulier
  (par exemple, bleu, jaune, vert, rouge). On réitére ces sessions
  pendant trois jours, avant de passer à la purification de la parole.

  \item \textbf{Parole.}
    \begin{itemize}

      \item \textbf{Création.} On commence par se visualiser empli de
      myriades de petites syllabes tibétaines \textsf{hung}
      \smallskip
      \begin{center}
        \tib \myhung
      \end{center}
      de couleur bleu foncé et irradiant de la lumière.\footnote{Ou
      entourée de flammes rouges?}  Chacune de ces syllabes, à chaque
      fois qu'elle est prononcée de façon longue, est chassée par la
      narine droite et finit par emplir la pièce, la maison, la ville
      et le monde entier.\footnote{Est\hyp{}ce que le corps se vide?}
      Il faut demeurer dans cette visualisation. Puis, en récitant les
      syllabes de la même façon, elles reviennent au corps par la
      narine gauche, descendent au c{\oe}ur, remontent à la gorge,
      puis emplissent à nouveau le corps. Cela habitue l'esprit à ne
      pas séparer soi\hyp{}même et le monde de la luminosité du
      \textsf{hung}.

      \item \textbf{Activation.} On prononce \textsf{hung} violemment
      pendant que la syllabe, de couleur noire et entourée de flammes,
      est projetée par la narine gauche. Tout objet sur la trajectoire
      est pulvérisé. Lorsque le monde extérieur est ainsi détruit,
      l'on demeure dans cette contemplation avant que ces syllabes ne
      reviennent au corps et le détruisent à son tour. Cela tranche
      l'attachement à la solidité du monde et du corps, qui est perçu
      comme corps illusoire.

      \item \textbf{Concentration.} On récite un \textsf{hung}
      normalement et le corps est visualisé instantanément sous la
      forme de la syllabe correspondante bleue foncée. Devant, la même
      syllabe de couleur rouge apparaît, de même taille que la
      bleue. À partir du \textsf{hung} bleu (le corps), des kyrielles
      de petites syllabes identiques (de la taille d'un empan) sortent
      comme des perles sur un collier ouvert et se dirigent vers la
      syllabe rouge et s'enroulent autour de celle\hyp{}ci à partir de
      la base vers le sommet, dans le sens horaire. Lorsque le
      \textsf{hung} rouge est entièrement recouvert (mais sa forme pas
      occultée), les petites syllabes reviennent à la queue leuleu
      vers le \textsf{hung} bleu (le corps), en se déroulant vers le
      haut. En chemin, elles deviennent rouges et entourent la syllabe
      bleue en descendant à partir du haut.\footnote{Dans le sens
      antihoraire?} Lorsque la syllabe est entièrement recouverte, on
      itère le même va\hyp{}et\hyp{}vient. Ensuite, la syllabe bleue
      (soi\hyp{}même), ainsi revêtue, se dilate jusqu'à devenir
      immense et l'autre fait de même. Les petites syllabes de
      revêtement se dissolvent et les deux syllabes immenses
      s'accolent sur leur côté et oscillent, comme en une danse, dans
      différentes directions. À la fin, toujours
      côte\hyp{}à\hyp{}côte, elles diminuent jusqu'à la taille d'une
      graine de sésame qui irradie le bleu et le rouge (correspondant
      respectivement aux essences paternelles et maternelles). Ce
      point lumineux se dissout ensuite et laisse place à un état
      mêlant intimement clarté et vacuité dans lequel on demeure. Cet
      exercice permet d'apprivoiser l'esprit rétif et dissipé.

      \item \textbf{Mouvement.} On se visualise sous la forme d'une
      syllabe \textsf{hung} blanche brillante de la taille d'une
      coudée. En prononçant lentement la syllabe, celle\hyp{}ci se met
      à s'élever au\hyp{}dessus du sol et à se mouvoir. Elle se rend
      d'abord dans des lieux familiers, de plus en plus lointains,
      sans rencontrer d'obstacles. Cet exercice purifie les habitudes
      (liées au mouvements du corps). On imagine ensuite que la
      syllabe se rend dans les Six Royaumes. Cela purifie les causes
      de renaissances dans ces lieux.\footnote{Et aussi la peur de
      renaître dans les royaumes inférieur et l'espoir de renaître
      dans les royaumes supérieurs?} La syllabe vole partout et tous
      les objets perçus en chemin deviennent radieux. Les Terres Pures
      sont aussi visitées et l'on imagine que l'on y reçoit les
      bénédictions des Éveillés primordiaux correspondant (le
      \emph{ma\d{n}\d{d}ala} des Cinq Familles est épuisé dans le sens
      horaire). Les objets de ces terres sont eux aussi purifé par le
      rayonnement intense de la syllabe sous la forme de laquelle l'on
      se visualise. Finalement, l'on rencontre Samantabhadra qui donne
      sa bénédiction à son tour.

    \end{itemize}

  \item \textbf{Esprit.} La purification de l'esprit (ordinaire) se
    fait au moyen de \emph{trekchö} et \emph{thögal}, mais l'on débute
    par un examen analytique pour déterminer l'origine de l'esprit, le
    lieu où il demeure et sa disparition. Suit alors un recueillement
    fondé sur les quatre attentions au corps (\emph{\'samatha}) pour
    obtenir un certain calme mental. On prend alors la position dite
    en Sept Points de Vairocana et on se montre attentif aux
    mouvements de l'esprit. Cela donne la possibilité de lâcher prise
    complètement. Le corps doit être immobile et détendu. (L'analogie
    traditionnelle est celle d'un corps sans c{\oe}ur.) Il doit être
    mis fin aux bavardages et il ne faut pas se laisser aller à penser
    au passé, au présent ou au futur. Il faut prendre garde à ne pas
    fabriquer la contemplation (par exemple, la limiter par des
    attentes ou pousser dans un sens ou dans un autre). Il faut
    demeurer dans la non\hyp{}pensée. Si, malgré ces instructions, le
    calme ne vient pas, on peut prendre un objet comme support (par
    exemple un caillou) et unir son esprit à celui\hyp{}ci, comme une
    absorption concentrée. On alterne alors concentration et
    contemplation.

\end{itemize}
Les étapes suivantes sont \emph{trekchö} et \emph{thögal}. La première
voie est la vue que tous les phénomènes sont purs et sans origine (ce
qui est la vacuité). La seconde voie repose sur les visions qui
s'élèvent spontanément de l'état naturel de l'esprit (ce qui
correspond aux moyen habiles).

\paragraph{Trekchö.}

Il faut rechercher un lieu isolé, s'assoir confortablement, jambes en
lotus, mains fermées (pouces sur la base des annulaires, autres doigts
repliés sur le pouce), le dos droit, les yeux fixés devant soi et l'on
expulse alors les impuretés par le souffle. (Trois fois trois fois,
selon la méthode habituelle.)  On demeure ensuite dans cette attitude.

Avant d'atteindre l'Éveil, au gré des pratiques, des expériences
extraordinaires peuvent être vécues mais elles ne sont pas
l'Éveil. Pour distinguer ces expériences de l'Éveil véritable, il faut
avoir une compréhension non intellectuelle de la vacuité. C'est ce
qu'apporte \emph{trekchö} et la présentation de l'état naturel de
l'esprit par le maître. Sans \emph{trekchö}, il n'est pas possible de
se relier correctement aux vision de \emph{thögal}, qui risquent alors
d'être saisies, réifiées et donc enfermeraient dans le cycle des
existences au lieu de libérer. La vacuité en question ici doit être
comprise comme étant identique à celle décrite par le M\=adhyamika.

\paragraph{Thögal.}

\emph{Thögal} comprend deux sortes de visions: celle des visions qui
sont produites par la médiation de la lumière du jour et celles qui
sont produites par la contemplation des ténèbres. Les premières sont
les formes de la vacuité, et les secondes sont les formes des sagesses
elles\hyp{}mêmes.

\paragraph{Thögal blanc.}

Les yeux mi\hyp{}clos et fixés sous le soleil, les \emph{bindu}
(tib. \emph{thigle}) apparaissent nombreux et mobiles, sur un fond
coloré dont la teinte varie selon l'état des canaux subtils. Il faut
poursuivre cette observation (le regard est tourné vers l'intérieur et
ne cherche donc pas à saisir quoi que ce soit) de façon assidue. Si
l'on travaille durant cette période, il faut se ménager quelques
instants pour fixer son regard sur un ciel sans nuages. Quatre lampes
sont liées aux \emph{bindu}.
  \begin{enumerate}

    \item \textbf{La lampe d'eau.} Elle représente les yeux, qui sont
      chacun reliés au c{\oe}ur par un canal subtil. Par ces canaux,
      les \emph{bindu} donnent l'impression d'être projetés dans le
      ciel, mais c'est en fait une vision intérieure (les \emph{bindu}
      ne sont pas des objets de la vision normale de l'{\oe}il). Se
      tenir dans la position dite du lion assis et fixer le ciel juste
      au\hyp{}dessus de la crête d'une montagne. Avec l'entrainement,
      des \emph{bindu} plus larges se manifestent, qui ressemblent à
      des yeux de poisson (donc vitreux, sombres et
      grisâtres). Concomitamment, la lumière devient plus intense et
      la vision ressemble à un brocard déployé entre les yeux et le
      soleil. Les petits \emph{bindu} vont se stabiliser, ce qui va
      révéler la lampe de la \emph{dharmat\=a}.

    \item \textbf{La lampe de la \emph{dharmat\=a}.} Elle est la
      vacuité elle\hyp{}même. Il faut bien comprendre que c'est la
      vacuité qui est la source des visions ici et non pas le
      ciel. Dans la position dite du \emph{rishi} (assis sur les
      talons, pieds joints par terre, genoux pressés sur la poitrine,
      bras enserrant les tibias et les mains touchant les pieds), l'on
      expérimente vacuité et félicité. Des \emph{bindu} sphériques,
      comme des arc\hyp{}en\hyp{}ciel quinticolores (correspondant aux
      couleurs des Cinq Familles) et ronds, comme des ocelles de paon,
      apparaissent sur le fond du ciel clair.

    \item \textbf{La lampe de \emph{rigpa} (tib.).} Le disque au
      centre des \emph{bindu} de la lampe de la \emph{dharmat\=a} est
      d'un blanc éclatant qui est le rayonnement de \emph{rigpa},
      l'esprit naturel qui luit en chacun, enfermé dans le vase de
      jouvence.

    \item \textbf{???} Les sphèrules quinticolores vont se dilater et
      en leur centre va apparaître une divinité (comme manifestation
      de \emph{rigpa}). Au début, la vision est floue et est dite
      correspondre à l'absence d'existence intrinsèque. Puis le corps
      divin apparaît clairement et surgit une grande joie,
      correspondant à la reconnaissance de la sagesse elle\hyp{}même.

  \end{enumerate}
  Avec l'entrainement, il y aura intégration des visions à la vision
  normale, jour et nuit, en une seule continuité,
  c'est\hyp{}à\hyp{}dire qu'elles n'apparaîtront pas uniquement durant
  les périodes de contemplation. L'expression «~Séparer \emph{rigpa}
    dans le ciel~» est glosée comme signifiant que \emph{rigpa} se
  manifeste comme s'il se projetait dans le ciel, donc séparément du
  pratiquant.

  Ensuite, \emph{rigpa} apparaît comme le \emph{ma\d{n}\d{d}ala} des
  Cinq Familles, c'est\hyp{}à\hyp{}dire qu'au lieu d'une déité seule
  ce sont les cinq Éveillés primordiaux en union avec leur parèdre qui
  sont manifestes (dans chaque \emph{bindu}). C'est la phase de
  maturation de \emph{rigpa}.

  Les visions ont atteint leur paroxysme et se dissolvent ensuite dans
  la vacuité. La pureté sans origine est reconnue et est infrangible.

  (L'état intermédiaire sera purifié par la reconnaissance des même
  visions de la lampe de la \emph{dharmat\=a} et l'Éveil sera obtenu.)

  À partir de cet instant, le corps grossier peut se résorber dans la
  luminosité fondamentale et réaliser le \emph{sa\d{m}bhogak\=aya} (en
  cette vie même). Puis l'esprit atteint l'Éveil complet lors de la
  dissolution.

  \emph{Trekchö} correspond à la vue supérieure du refuge, dite
  non\hyp{}duelle. \emph{Thögal} est l'aspiration à l'Éveil
  (\emph{bodhicitta}), à l'activité bénéfique à autrui.

\paragraph{Thögal noir.}

On recherche un lieu calme et auspicieux. On s'assoie en posture du
lotus, le dos bien droit et l'on s'amuït. En pressant doucement les
globes occulaires, l'on fait jaillir la luminosité
fondamentale.\footnote{Soulever la paupière en pressant le globe?}
D'abord une lumière rouge se déploie, puis blanche et enfin une grande
obscurité. Par la suite, les couleurs des Cinq Sagesses se manifestent
clairement. Ce qui se trouve au\hyp{}delà de la cellule peut être vu
par moments et ainsi les notions d'intérieur et d'extérieur
s'estompent. La saisie mentale se relâche. La vacuité est perçue, mais
incomplètement (par exemple, on peut oublier la présence des murs et
s'y cogner). Puis le sentiment d'absence d'obstacle se parachève et,
après trois mois, le \emph{sa\d{m}bhogak\=aya} est
réalisé.\footnote{Est\hyp{}ce à dire qu'un corps subtil, qui peut se
  mouvoir librement, est produit?} Tous les phénomènes extérieurs sont
perçus comme illusoires.

Pour parvenir à ces accomplissements, trois aspects doivent être
considérés, en plus du choix du lieu idoine.
\begin{itemize}

  \item \textbf{Lieu.} Choisir un lieu calme et isolé, qui est
    agréable. La cellule doit être parfaitement isolée du jour, au
    besoin en doublant les cloisons. Plusieurs rideaux épais ferment
    l'accès. La pièce est aspergée d'eau safranée. On pose le coussin
    de méditation dans une position un peu surélevée (pas à même le
    sol), par exemple sur un lit aéré.

  \item \textbf{Corps.} On prend la position dite du \emph{rishi}, à
    laquelle on adjoint un bâton par devant, pour soutenir le
    menton. Le pied droit est un peu en retrait et l'autre légèrement
    en avant. Il convient de se faire servir la nourriture à heures
    fixes et il est préférable que l'assistant soit lui\hyp{}même
    bouddhiste. On reste alors concentré, abandonnant toute activité.

  \item \textbf{Parole.} On coupe court au bavardage mental et verbal,
    même les prières.

  \item \textbf{Esprit.} On tranche toutes les pensées, mondaines ou
    non, même la quête de la vacuité. Il est bon de commencer tôt le
    matin.

\end{itemize}

\paragraph{Yogas.}

Des yogas et des exercices physiques doivent être pratiqués dans le
but de dénouer les roues de canaux subtils (\emph{cakra}). On commence
par s'assoir en position du lotus et purifier le souffle selon la
méthode habituelle (trois fois trois fois) et on poursuit par la
méditation dite du vase. Ensuite, reposer les mains l'une sur l'autre
puis les claquer trois fois, à gauche, à droite et
devant.\footnote{Une fois dans chaque direction?} Puis basculer le
corps pour dénouer la roue du c{\oe}ur. Ensuite, la pointe de la
langue à l'intersection du palais et des dents de devant en haut, on
tourne la tête trois fois, trois fois encore et puis une dernière
fois. Cela dénoue la roue de la gorge. Secouer ensuite la tête devant
et derrière trois fois, dans le but de dénouer la roue de la tête. Les
exercices physiques incluent des sauts, comme le trampoline. Puis
respirer par les deux narines très lentement.

Attention! La méditation du vase doit être maîtrisée avant de débuter
la retraite, sinon elle présente un risque sérieux d'épuisement.

D'abord on prend la position du \emph{rishi} avec la ceinture et le
bâton. Ce dernier doit être aspergé d'eau safranée, être bien lisse au
niveau du support du menton et, sur cette partie haute, il est
auspicieux que soient gravés six pétales de lotus et qu'il soit
parfumé à l'eau safranée. On regarde intensément vers le haut en
pressant légèrement les globes occulaires.
\begin{itemize}

  \item \textbf{Instructions.} Le corps est composé des quatre
  éléments et l'un domine chez chaque personne. Si c'est l'élément feu
  qui domine, il est bon de se visualiser immobile sur une mer bleue
  calme. Pour approfondir la méditation, on se visualise comme une
  montagne enneigée entourée par une mer agitée. Si l'on expérimente
  le vide, il faut imaginer son corps sur une mer laiteuse et calme.

  \item \textbf{Étapes.} Le \emph{tantra} de \emph{L'{\OE}il Unique de la
  Sagesse} conseille les pratiques suivantes, étalées sur une semaine.
  \begin{enumerate}

    \item On se visualise comme un ballon gonflé. On prend la position
    du \emph{rishi} et l'on visualise l'{\oe}il de la sagesse sur les
    cinq points de la position (deux pieds, deux mains et la
    tête). Puis l'{\oe}il est visualisé sur la roue de la tête. Ceci
    pendant un jour et une nuit. Cela permet de syntoniser la
    respiration, \emph{rigpa} et la conscience.\footnote{Consciences
    sensorielles ou toutes les consciences selon le Cittam\=atra?}

    \item On médite un {\oe}il de sagesse au c{\oe}ur, qui est le
    réceptacle des cinq sagesses. Les yeux physiques regardent
    intensément droit devant, en haut, en bas, à gauche et à droite,
    sans tourner le buste. Ceci pendant un jour et une nuit.

    \item On se visualise comme une déité courroucée, qui est la
    manifestation de la fierté et de l'intrépidité. Les yeux
    ordinaires sont considérés comme des yeux de sagesse pacifiques,
    regardant vers l'avant. Sur l'occiput se trouvent deux autres yeux
    de sagesse courroucés, regardant vers l'arrière. On visualise un
    {\oe}il de sagesse \emph{supplémentaire} sur le front (pacifique)
    et un autre sur l'occiput (courroucé). Les deux paires d'yeux
    initiales tournent ensuite leurs regards l'une vers l'autre, et
    les yeux supplémentaires font de même. Ceci pendant un jour et une
    nuit.

    \item Les yeux visualisés au point précédent sont transformés en
    cymbales qui sont projetées répétitivement vers leur
    vis\hyp{}à\hyp{}vis. Elles s'entrechoquent en dégageant des
    éclairs brillants, des arc\hyp{}en\hyp{}ciel et des Terres
    Pures. Ceci pendant un jour et une nuit.

    \item Dans la position du \emph{rishi}, on regarde vers le
    c{\oe}ur qui est marron, ouvert sur son sommet et dont le bord
    supérieur est vert. Il contient un {\oe}il de sagesse qui regarde
    vers le haut. Les six yeux de la tête regardent alors vers le
    bas. Ceci pendant un jour et une nuit.

    \item Le palais de conque est dans le cerveau. La conque en
    question présente son ouverture vers le bas. Elle contient un
    {\oe}il de sagesse qui fixe l'{\oe}il dans le c{\oe}ur. Ces yeux
    se projettent l'un vers l'autre, répétitivement, en émettant des
    rayons comme précédemment. Ceci pendant un jour et une nuit.

   \item Toujours dans la position du \emph{rishi}, les yeux
   ordinaires regardent vers le haut, ce qui favorise la contemplation
   de la pureté primordiale pendant un bref instant. Avec la
   répétition, il devient possible de demeurer stable dans cette
   contemplation. Ceci pendant un jour et une nuit.

  \end{enumerate}

\end{itemize}

\end{document}
