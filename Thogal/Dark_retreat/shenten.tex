%%-*-latex-*-

\documentclass[11pt,twocolumn]{article}

\usepackage[british]{babel}    % British English
\usepackage[T1]{fontenc}       % Required for hyphenation
\usepackage[utf8]{inputenc}    % UTF-8 encoding
\usepackage{hyphenat}
\usepackage{microtype}            % Microtypographic enhancements
\usepackage[charter]{mathdesign}
\usepackage{graphicx}

\title{A Short Dark Retreat of the Leap-over}
\author{Christian Rinderknecht}

\begin{document}

\maketitle

\noindent
[I have written this short account at your request, dear friend. I
  hope it will benefit you in some way. Even though the particulars
  are expected to vary according to people's abilities and
  characteristics, they might be a source of inspiration. For context,
  solitary retreats in total darkness are part of some cycles of
  teachings in the Dzogchen tradition of Tibet (`The Great
  Perfection').]

The \oldstylenums{22}nd of December \oldstylenums{2025}, I reached
Shenten, a \emph{domaine} in rural France, close to Saumur, which was
repurposed twenty years ago as a Bön retreat centre, with residing
lamas all year round. The Bön is a religious tradition of Tibet that
is hardly distinguishable from some other Buddhist schools: different
scholastic, texts and lineages, but the same view and similar
methods.\footnote{One difference with the Buddhist syllabus is the
study of poetry, which goes to my heart.}

A resident practitioner had me sign a document listing the possible
negative effects on my mental health of being secluded in the dark for
a long period of time, and requiring that I agree to a dark retreat
after this fair warning and under my own private responsibility. This
is really important, as I have noticed over many years how Buddhist
institutions in the West tend to attract people with all sorts of
mental illnesses. One contributing factor is that too many
psychotherapists recommend their patients to practise some form of
Buddhist meditation. I could expand considerably on this topic, the
relationship between psychotherapy and Buddhist meditation (or the
secular and misguided so-called mindfulness), the widespread use of
psychotherapy in the West (to the point that when someone says
`therapy' it means `psychotherapy') without ever mentioning the
possible negative outcomes, or its possible uselessness etc. But,
given that isolation in a cell is the worst punishment in prisons, and
that I was not only going to do it willingly, but also in total
darkness, I think the warning and disclaimer were more than fair: they
were absolutely necessary.

After dinner, the lama in charge of the retreat, Geshe\footnote{An
academic degree in the Bön tradition --- and others ---, translating
as `virtuous friend'.} Geleg, a tall, warm man with a high reputation,
sat with me at the dining table. He told to me in English that it will
be crucial to relax. I understood that in the Dzogchen sense, that is,
to remain as much as possible in the knowing of the primordial ground,
\emph{rigpa}. Rigpa has two aspects (when not considered a process for
exposition's sake): \emph{primordial purity} and \emph{spontaneous
presence}; the former is, in rough terms, emptiness inseparable from
wisdom, and the latter, manifestation of that potential wisdom into
displays of clear light visions. The cultivation of primordial
awareness is achieved by practising \emph{trekchö} (`Breakthrough (the
solidity of the ordinary mind)'), and the knowing of the spontaneous
presence require a practice called \emph{tögal}
(`Leap\hyp{}over'). The Leap\hyp{}over can be practised with daylight,
moonlight, even candlelight, and also in total darkness, but
familiarity with \emph{trekchö} is needed for its progress and
success.

So Geshe Geleg reminded me that the traditional duration of a dark
retreat is seven weeks,\footnote{A number corresponding to the time
someone is supposed to migrate from death to rebirth.} but modern,
busy life calls for more flexibility. He told me that there are no
specific instructions. Usually the retreatants would simply practise
what they practise daily. I was surprised, because the dark retreat in
the Yangti Nagpo cycle of the Nyingma school comes with complex,
bizarre yogas, as well as strenuous physical exercises meant to
exhaust the discursive mind.

Geshe Geleg added that, should any inner or outer obstacle arise (the
latter I understood refers to some mischievous, mundane spirit), he
recommended a simple tantric practice. Visualise in front of me the
Tibetan syllable \emph{so}, which he drew on an envelop, coming from
my heart, and imagine it in a bright, pulsating red light. Then repeat
the \emph{so} sound, while the bright syllable multiplies and fills
the space entirely. Next, I should imagine myself as a wrathful deity
in red, without focusing on the external syllables anymore. He used
his arms to show me a threatening posture often found in
\emph{thangkas} (the religious paintings hung on walls). Usually, when
practitioners visualise themselves as a wrathful deity, the goal is to
cut at the root attachment to the ego. But in this context, the goal
is to scare away evil spirits, as they naturally can see the contents
of our mind. Then Geshe Geleg told me that I should imagine the
syllables entering my crown wheel (the \emph{cakra} a few centimetres
above the head), descending into the central channel where they pile
up, then filling up my entire body. Finally, I should dissolve the
entire visualisation into the vastness of space.

Armed with those simple instructions, I walked behind him in the field
to the retreat house. It was already dark, as we were far from any
town. He was holding a stick of incense, and I did not know if that
was relevant to the retreat, or if he simply needed some light. He was
singing softly, and I asked him: `Geshe-la, you seem happy.' Without
looking at me, he shrugged and replied: `What else is there to do
while we remain in \emph{sa\d{m}s\={a}ra}?'

The wooden house was built on stilts (as tradition mandates). I
followed him over the threshold, and left my shoes next to two other
pairs of women shoes. So I had neighbours and I will be the only one
snoring... With this happy thought I entered the cell, a tiny but
cosy room with a window, a bed, a meditation cushion, a desk and a
bathroom. More unusually, the window had both (inside) blinds and
(outside) shutters, and a box in an external wall with two doors, one
opening from the outside, one from the inside. The purpose of that box
is to provide food without letting light in, but also to talk once in
a while with Geshe Geleg when he comes to check in and counsel. But by
far the most perplexing was the presence of a sheet of paper and a
pen. Would have been funnier if the paper had been black, though.

He asked me to sit on the cushion; he used the bed. He was going to
recite a prayer in Tibetan to Tapihritsa, the great master of the
past, whom he instructed me to visualise in front of me, dispensing
his blessing. Afterwards, Geshe Geleg told me: `Your moment has
come. Eat your food. Do not wash often, as this would allow the wind
element in your body to escape. You should practise at least four
times per day. I will come to speak with you regularly.' Back on the
threshold of my room, he flipped a hidden electric switch, and I heard
the door close, but no goodbye.

Yet the room never went pitch black for me.

Immediately, white lights shone forth. Not bright, but veiled like
street lights and car headlights seen from behind a curtain in a dark
apartment. Just white, slowly moving and nondescript shapes slowly
changing and moving before disappearing. This gave me the false
impression that there was light in my cell, and shadows. I was not
surprised by the displays of white clear light, as I was already
familiar with them, but by how suddenly they manifested, without
transition. Of course, this is obvious if we understand that they were
already shining before the cell was plunged into darkness. A dark
retreat forces the retreatant to focus on what is there, hiding in
plain sight. Now, not everybody has the same experience right off the
bat: it depends on karmic predispositions and one's achievements in
this lifetime. But everyone \emph{can} see them, because they are the
manifestation of the primordial ground, better recognised by prior
training in the Breakthrough.

I remember the first time I saw the clear light shine as an adult. At
the end of my twenties, I had started to practise \emph{zazen}, the
sitting meditation of the Japanese Buddhist school known as Sōtō
Zen. It is a meditation without object: one focuses on a rigid
physical posture, and on the breathing, while facing a white
wall. After a couple of years of regular practice, after some time in
meditation, when the mind was calm, I used to notice on the white
wall, or my bathroom door, a change in the lighting of the room, then
yellow light, then blue,\footnote{The five colours of Buddhism are
white, blue, yellow, red and green. This is not a random coincidence:
the tradition chose those because practitioners actually were seeing
them.}  the deeper the contemplation went and the more I imagined my
breath enter the central channel and descend into my heart
wheel.\footnote{The Zen uses an enigmatic instruction: `Have the
breath descend even on expiration.' At least, it was my interpretation
that that refers to the \emph{internal} breath (a so\hyp{}called
subtle wind), not the external breathing of air.} The change in
lighting was actually a white clear light shining. This prompted me to
question my Zen teachers but they either did not understand or
recommended me to ignore this extraordinary phenomenon, lest it would
become an obstacle. I chose to leave and find a teacher in Tibetan
Buddhism who, after a moment of surprise, assured me that my
experience was actually a very positive sign of progress. My point
here is that I did not practise the Breakthrough to plough the field,
but \emph{zazen}.

What really surprised me was how easy it was for me to find my
belongings in the dark room, or walk in the room. Darkness turns a
concrete experience of the world into an abstract one, and I am rather
good at mental abstraction and memorisation.

During that night, I had at least three dreams in black and white. In
the one I remember, I knew that I was supposed to be in a dark room,
but I kept seeing white light, which I assumed was coming from
outside, and so I kept looking for the supposed leaks, confused at
their moving location, like a cat chasing after the spot on the ground
made by a laser pointer. This confirmed, if need be, that the visions
were continuous, self\hyp{}arising, not produced by the ordinary mind
as in tantric visualisations. Those visions are not external: they
take place in the heart wheel, and are perceived by the brain as
taking place in front of us.

The following day was the first day. The emergence of white light was
less intense, and I slept almost all day.

The second day, I started to see impure visions. I call them impure in
opposition to the previous visions, because the impure visions are
figurative, not abstract shapes. Nevertheless, I knew that they were
the same clear light emerging spontaneously from the primordial
ground, but now the ordinary mind, exactly like in a dream, would
\emph{interpret} them into something familiar, corresponding to an
emotional state for instance, instead of letting them arise and
disappear on their own. Fortunately, I was not taken aback by what
could be construed as a reversion (from pure to impure). It was indeed
my experience of meditating without object, at least after some months
or years of experience, that the first moment of contemplation can be
deep but unstable, and then I would have to cut through a jungle of
paradoxical injunctions, of doing something without doing anything,
for example.\footnote{Another school of Zen, the Rinzai, verbalises
such paradoxes, or \emph{kōan}, to exhaust the logical, rational mind,
and hopefully experience a clearing beyond, a breakthrough.} Now I am
reminded of a quote from a Chan\footnote{The ancestor of Zen in
China.} master who used the following analogy: when we want to
traverse a field of thorny bushes, the first step can be relatively
easy, but not the subsequent ones. I do not remember the context, but
this analogy could fit within the famous historic antagonism between
the Northern and Southern Chan schools, summarised by: Is
Enlightenment sudden or progressive?  I personally like to think
instead of meditators as pearl hunters who plunge from a cliff into
uncertain waters: they can easily go very deep at first, have a
glimpse of the secret life down there, but they are inexorably drawn
up to the surface, and going back down requires skill. Another analogy
I read in a teaching by Trungpa Rinpoche was: when we approach a
painting on a wall, we first \emph{see} it, and, as we get closer, we
\emph{look} at it. The context was to contrast two kinds of wisdom:
\emph{jñāna} and \emph{prajñā}, respectively. This seems
counter\hyp{}intuitive at first, because we are used to think that
learning, therefore knowledge, requires gradual effort. This is not
the view of Dzogchen, where the unveiling would be a more accurate
metaphor. (Socrates would perhaps agree with this `remembering' too.)

Back to the impure visions of the second day. They would be vast
scenes (think IMAX cinema), white, yellow and black. Each time, at
first, a pure vision would arise, then it would transform into a
static scenery before disappearing, then another would arise
---~without cease. The transition from pure to impure felt like an
optical illusion: even if we are prepared to see an optical illusion,
no matter our knowledge and resolve, for a fraction of a second we see
an abstract shape, but once the brain tricks itself, it cannot revert
back to the original vision. And so my visionary experience of the
second day felt like a repeated fall into a profound state of
forgetfulness and ignorance of reality, of the process that lead me to
this point. It was not as if an impure vision was part of the whole
and my sight was confined to it. No. It was more that the turning into
an impure vision was the visual display of an inner fall into
ignorance. Those visions arise from the primordial ground, they are
not separated from us. They are beyond morality too: they are not
evil, not even because they are impure; instead, they are impure
because we have not exhausted yet the seeds of conceptuality, and what
was sown will inevitably ripen, in this life or the next ---~better
this one. This confusion about reality, the duality between subject
and object is typical of dreams, where the ignorance is even deeper,
as the sense of being an observer is lost (except during lucid
dreaming, as in dream yoga), like in an hallucination. Anyway, it was
difficult for me to attend to those impure visions because they were
growing in number and frequency, all the while I was feeling very
confused ---~uncontrollable and innumerable thoughts would be depicted
in front of my eyes at high speed.

Geshe Geleg came knocking. He asked about my dreams, so I did not tell
him what happened the first evening, that I did not need to have signs
of progress in dreams, but I did tell him that I was having lots of
dizzying thoughts. He told me that this was a positive sign, as it
means that the mind is calming down enough to start noticing the
plethora of thoughts that are normally constantly simmering, and now
express themselves, and by themselves. He told me to attend to the
visions, but never engage with them, just let them be. Finally, he
told me: `Do not create the past, do not fabricate the future. Don't
forget to enjoy it. This is a unique occasion to know yourself.'

The third day, the runaway train of thoughts slowed down by itself,
and now the common denominator of all the impure visions was that they
were all peaceful \emph{to me}. For me, since childhood, peace can
only be found away from people, in nature. And, since I have a
fondness for the night sky in the country side, the moon, rivers,
lakes, clouds, tall forests, stone villages, majestic crags, snow, I
had two days and nights of such peaceful, impure visions continuously
appearing. But, despite their peacefulness, those images appeared
frozen, veiled, and I felt compelled to like them. Their scale and
depth seemed tremendous at times.

I started to undertake my four daily sessions of practice. I always
liked \emph{guru yoga} and purification. Unfortunately, it would take
us too far afield into these here. I normally choose the Bön guru
yoga, but keep the Nyingma purification. On the one hand, I quickly
realised that guru yoga would not only not disrupt the visions, it
would stabilise them, or, equivalenly, my contemplation of them,
because of the focus on the central channel. On the other hand, the
purification did not work well, and I wondered why. It occurred to me
that the main difference between them was that guru yoga was barely
figurative (a white tigle), whereas the purification was much more
elaborate (Vajrasattva), more tantric in a way. So I simplified it and
it started to fit with the self\hyp{}arising of the visions. Again,
this shows, if needed, that the visions are natural, uncreated,
whereas tantric visualisation are forcefully imagined and projected,
hampering the emergence of the clear light. In passing, since I did
not want to disturb my neighbours, I would whisper my mantras, which
made me discover that I could use the inspiration to pronounce the
syllables, yielding a continuous recitation.

I would intersperse practice and sleeping. I noticed that simply lying
on the bed was not good, as I would drift into thinking about the past
or wonder about the future. Fortunately, I had in me the discipline to
get up, even in the middle of the night in case of insomnia, wash my
face and sit on the cushion until I felt tired enough to actually fall
asleep. That night, I had a dream that was a sign of progress with
purification: I was looking at my right foot and ankle, and I would
pick up insects that had been sucking my blood. It was neither scary
nor painful, more a calm relief. The phasm\hyp{}like\footnote{Pun
intended.} insects would come out easily in the pinch of two fingers,
with a bag of blood attached to their bodies. I remember reading in a
book by Tenzin Wangyal Rinpoche how he had a similar dream when
practising the purification in the preliminary tantric
practices.\footnote{The \emph{ngöndro} is meant to establish stability
in the practitioners, to bring to the fore glimpses of insight, to
enable them to benefit from other teachings. It is like the making of
the foundation of a building.} So purification was finally working in
the dark.

The fourth day, I realised that I should have not exclusively
practised on the meditation cushion: my lower back was aching a
lot. Even lying down, whatever the position, was very painful. I also
had a cold and a piercing headache that turned into a migraine that
lasted twelve hours. Zazen teaches us to remain still no matter
what. If the nose runs, let it be. If something hurts, let it hurt;
actually, observe the feeling without interfering. Who is hurting?
When beginners would tell me of such and such cramp or physical
discomfort after a session, I would tell them: `How lucky you are to
have such distractions. I only had my messy mind to contend with.' So
I believe that that training helped me to keep going without
struggling ---~a needless fight only adds to the pain. That day was
rather dull and dim. The vast and peaceful visions had become rare and
nothing seemed to happen. This was also a potential source of worry,
therefore a threat to the entire process I committed to. So I took
everything in my stride. This is also why faith is essential.

Sitting on the chair, large impure visions rose again, but this time
they all had in common being aggressive imagery, scenes of war, of
street fights, of town destructions; an army would be walking in a
forest at night, people would fight, kill or threaten each other
etc. It is perhaps worth mentioning that those scenes were not aimed
at me, the me sitting on the chair, but rather had to do with
something in me that was coming out. This realisation was as
unsettling as it was unexpected. Moreover, there was a second kind of
impure visions, albeit rarer than that of violence: depictions of
couples kissing, sometimes the woman would have her eyes open and
stare at me while kissing the man, or there would be an attractive
naked women bathing on the other side of a wide river, and many other
uncountable sultry scenes ---~all rather tame and chaste for today's
standards.

This lasted three days. When Geshe Geleg came to advise, I told him
about the Everests of anger manifesting themselves into shocking
visions. He said that these were karmic seeds being purified: I would
not have to be slave to them anymore, and I just had to let them
express themselves loudly and they will pass, like the peaceful
visions. He said (in substance): `People wrongly believe that karma
can disappear without traces at this stage in the spiritual path. Do
not engage with the visions. Remember to dwell in the recognition of
\emph{rigpa}. To bring it about, to avoid getting stuck in the
contemplation of impure visions, in order to regain the freshness of
the open space, focus on the dissolution phase of guru yoga: give it
100\%! And don't forget to enjoy this retreat.' I asked him whether I
could stay a day longer, as I wanted to see what was in store after
this phase, and I had planned enough days. The answer would come back
with a meal the next day: yes.

It turns out that I was rather mistaken about the dissolution phase of
guru yoga... I believed that Tapihritsa had to descend into my central
channel down to my heart wheel, and then my body had to contract and
merge with his body and transform into a drop of pure white light
(\emph{thigle} in Tibetan), which would then suddenly expand to
infinity, but this had a different effect than returning to
\emph{rigpa}: I was sensing a sudden great heat in the heart region,
rapidly expanding to my whole body; this would happen at the same time
that I would lose feeling of my body while being acutely conscious,
and feel that I was exiting my body. Just before that I would worry
about my breathing (as I would not feel my body) and forcefully
breathe in, rather perplexed. I could not check in my booklet what the
dissolution phase was, so I imagined something softer.

After three days, I had a couple of dreams that confirmed an
astonishing transformation I was barely starting to fathom. The first
dream I remember is me at my second job in my career. My boss was
aggressively accusing me of some exaggerated or imaginary wrongdoing,
as often used to really happen, but, this time, in what had started as
one of my many recurring nightmares, I laughed and explained to her
that she was mistaken in her accusations. Of course, this concluded
the dream, as, in real life, she never got to hear that from me. The
other dream is stranger. I had a business partner who, without my
permission, used the company's money, our money, to invest in what
resulted to be a scam, and I lost everything. Instead of feeling
betrayed and desperate, I went to the scammers. They presented
themselves as an investment fund, with an office within a large
building. They were a handsome couple in love. The woman was smiling
at me, but denied any wrongdoing. Instead of confronting them, I spoke
with the manager of the building, a beautiful blonde with straight
hair. I told her what happened and she believed me, but could not help
me without proof. She nevertheless thought that she should secretly
keep an eye on those scammers to avoid others falling into their
trap. Meanwhile, I decided to put my (actually nonexistent) skills as
a computer hacker to access the scammers bank accounts and realised
that they had defrauded many people. Moreover, there was not enough
money to refund them, so, instead of recovering my money, I decided to
transfer to them what was available. Then the scammers realised what I
had done and went after me, inside the building. I had had to bypass
some security to access the bank server, and they knew that and were
coming after me with ill intent. I tried to open some office doors in
order to hide, until I succeeded to get into an empty office
room. Suddenly, the office manager also entered the room. Surprised, I
told her: `This is not what it looks like.' I meant that I was not
stalking her, and she understood my meaning. She replied, with a
smile: `And what if it were?' She approached and kissed me ---
\emph{Scene}. When I woke up, I realised that all my bottled up anger
at being treated unfairly for decades had evaporated. I just could not
find it anymore! I would conjure a painful memory that would normally
make me angry or resentful, but I would not feel \emph{that} anger,
only peace. Furthermore, I also understood what the sensual visions
meant. I always used to find attractive and interesting women to be
rather inaccessible; I always felt jealous of happy couples ---~ gone
as well. Completely gone. I tested my theory by then summoning rather
painful memories of past rejections, of being used, abused and
betrayed by women; memories of beautiful and intelligent women who
were being spoken for: no hurt feelings, no self\hyp{}pity would come
up!

Interestingly too, on the sixth day, I started to see pieces of pure
visions amidst the fading angry ones: those were like stars in the
night sky. They were essential drops, known as \emph{thigle} in
Tibetan, the next stage of the visionary experience according to the
teachings.

The seventh day, the visions of aggression had completely vanished,
leaving the stage to lots of discursive thoughts. I was also
contemplating peaceful impure visions again and more thigles on the
background or in front of them (the latter more difficult to
discern). At the same time, I was thinking entire paragraphs about
sundry topics. Fortunately, it felt easier to interrupt those
thoughts, but the flow of them was overwhelming. The thigles slowly
started to appear clearer and in greater numbers: they were actually
arranged in a square grid, and I felt that there were other,
innumerable similar arrays behind. I also started to see some large
thigles: they appeared close like disks either white, red or blue
(rarer). At some other moments, I would only see circles of said
colours. The difference is due to the depth of the contemplation:
ultimately, all thigles are of the same nature, they are the (hobbit)
doors to the arising of the deities. So I would see on the lower part
of my vision field\footnote{In a manner of speaking, since this was
taking place in my heart wheel.} stacks of thigles of the same
colour. They were not seen clearly yet, though. I needed more time and
those discursive thoughts were very distracting, as I love writing. I
was writing an entire book in my head, like the man condemned to death
in Borges' short novel \emph{The Secret Miracle}!

This continued the eighth day, which was unfortunately the last of my
retreat. Geshe Geleg entered my room in the evening and read a closing
prayer. The shutters were unlocked from the outside, and the blinds
rolled up. I was to stay in the room until noon the next day, as
tradition mandates. The natural morning light will help my eyes to
recover their normal sensitivity. I had anticipated a disruption in
the production of melatonin in my body, so I had asked the cook to put
on my tray a pill with every dinner ---~I didn't want to be jet-lagged
without moving. Nevertheless, it took me some time to use my eyes
while I did things, and when I did I felt slightly dizzy, but
everything went back to normal in time.

It is my intention to return and do a longer retreat of three weeks,
as I want to see beyond the eighth day. The therapeutic
side\hyp{}effect of this retreat was entirely unexpected and
welcome. It achieved results where, I am certain, talk therapy would
have not. Psychology is like dealing with the foam of the waves; a
dark retreat is plunging into the depth of the ocean. It would be a
mistake to envisage Buddhist practice as a therapy, even if
therapeutic side\hyp{}effect do occur ---~just like paranormal
abilities can arise. Motivation is key to the result.

Now, as a postlude, I would like to gather some random elements that
otherwise would have slowed down even more the narration.

Firstly, a curious vision. I remember seeing it twice. Both times, it
was a yellow pure vision (therefore abstract). Like yellow water
spurted from a pierced balloon, a group of what looked like Tibetan
letters were ejected from a vision. They were not Tibetan letters, but
seemed familiar nonetheless, and it took me a few moments to remember
where I saw something similar. Twenty years ago, I came across an
example of a so\hyp{}called \emph{\d{d}ākinī script} from a revelation
of the kind `Liberation upon seeing' (see
figure~\ref{dakini_script}). At the time I thought that the meaning
was that those who saw the script \emph{in print} would hopefully be
liberated. After my experience, I now think that the actual meaning is
that whoever \emph{sees this script in a Leap\hyp{}over vision} will
likely be liberated. I guess I will see.
\begin{figure}
  \centering
  \includegraphics[width=0.4\textwidth,height=5cm]{dakini_script.jpg}
  \caption{`Liberation upon seeing'}
  \label{dakini_script}
\end{figure}

Secondly, what about the effect of the exhaustion of the peaceful,
impure visions, at the beginning of the retreat? My take is that my
longing for living in remote, somewhat isolated places\footnote{I did
went for a job interview in the Faroe Islands.} has diminished, but it
is difficult to test. And what of the discursive thoughts? I did not
conclude that stage, but perhaps my obsession for writing will perhaps
be somewhat less spellbinding. Clearly, this is not yet the
case. (Today, in writing this in one sitting, I forgot lunch and
dinner.)

Lastly, I would like to close with an excerpt from the \emph{Vajra
Heart Tantra} by Düdjom Lingpa:
\begin{quote}
  \it Now as for the stages of the main practice, at first you
  determine the ground by way of the Breakthrough, then the initial
  moment of impure consciousness emerges in the aspect of an object, a
  subsequent conceptualisation fastens onto it, and delusion sets
  in. Now, in contrast, in the Leap\hyp{}over, the initial moment of
  consciousness is transformed into an appearance of clear light, and
  by experiencing the very nature of consciousness, all impure
  appearances dissolve into the absolute nature and vanish. Knowing
  how that occurs is the indispensable, sublime point of the
  Leap\hyp{}over, so recognise it!

  If you do not recognise this vital point, however much you meditate,
  you will go astray on the path of dualistic grasping, and you will
  not progress along the grounds and paths of liberation. Thus, once
  you have truly realised the manner in which the whole of
  \emph{sa\d{m}s\={a}ra} and \emph{nirv\a={a}\d{n}a} is none other
  than your own appearances, finally all mental states and appearances
  of the impure cycle of existence will forcefully be transformed into
  displays of the clear light, reality itself. So this is the
  practical guidance on the great transference. By truly recognising
  the entrance to this path with the wisdom of realising
  identitylessness, originally pure reality\hyp{}itself, beyond mental
  investigation, the absolute nature free of conceptual elaboration,
  will be experienced with the eye of expansive wisdom. Unlike
  nebulous, obscure meditations and intellectual fabrications, with
  the eye of wisdom you directly see the precious, spontaneously
  present absolute nature, the reality\hyp{}itself of the expanse of
  clear light.
\end{quote}

\end{document}
