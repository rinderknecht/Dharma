%%-*-latex-*-

\documentclass[11pt,twocolumn]{article}

\usepackage[british]{babel}    % British English
\usepackage[T1]{fontenc}       % Required for hyphenation
\usepackage[utf8]{inputenc}    % UTF-8 encoding
\usepackage{hyphenat}
\usepackage[charter]{mathdesign}

\title{The Vajra Heart Tantra\\
Excerpt on T\"ogal (Leap-over)}
\author{D\"udjom Lingpa}

\begin{document}

\maketitle

Now as for the stages of the main practice, at first you determine the
ground by way of the Breakthrough, then the initial moment of impure
consciousness emerges in the aspect of an object, a subsequent
conceptualization fastens onto it, and delusion sets in. Now, in
contrast, in the Leap\hyp{}over, the initial moment of consciousness is
transformed into an appearance of clear light, and by experiencing the
very nature of consciousness, all impure appearances dissolve into the
absolute nature and vanish. Knowing how that occurs is the
indispensable, sublime point of the Leap\hyp{}over, so recognize it!

If you do not recognize this vital point, however much you meditate,
you will go astray on the path of dualistic grasping, and you will not
progress along the grounds and paths of liberation. Thus, once you
have truly realized the manner in which the whole of sa\d{m}s\={a}ra
and nirv\a={a}\d{n}a is none other than your own appearances, finally
all mental states and appearances of the impure cycle of existence
will forcefully be transformed into displays of the clear light,
reality itself. So this is the practical guidance on the great
transference. By truly recognizing the entrance to this path with the
wisdom of realizing identitylessness, originally pure
reality\hyp{}itself, beyond mental investigation, the absolute nature
free of conceptual elaboration, will be experienced with the eye of
expansive wisdom. Unlike nebulous, obscure meditations and
intellectual fabrications, with the eye of wisdom you directly see the
precious, spontaneously present absolute nature, the
reality\hyp{}itself of the expanse of clear light.

To practice these instructions, at the outset you must firm up your
posture, for if this is not done, the absolute nature, bindus, and the
vital energies will be dispersed in all the channels and elements of
the body, and they will not manifest. As an analogy, if a snake is not
squeezed, its limbs will not become evident, but if it is, they
appear. The posture is accordingly of tremendous importance.

First, the lion's posture is as follows. Join the soles of your feet
in front of you. Plant your vajra\hyp{}fists on the ground between
your legs, and look up into the sky. That is the dharmak\a={a}ya
posture and gaze. For the sa\d{m}bhogak\a={a}ya posture, plant your
knees and elbows on the ground and support your cheeks with your
palms. Point the soles of your feet outward, and gaze directly in
front of you. However, if appearances of the clear light do not
manifest, alternately run your gaze to the left and right and up and
down. Rest your gaze wherever those appearances are most clear. For
the nirm\a={a}\d{n}ak\a={a}ya posture plant the soles of your feet on
the ground, press your chest against your knees and clasp your knees
with both hands while interlacing your fingers. Straighten your spine,
and gaze downwards.

Here is the significance of those postures. With the dharmak\a={a}ya
posture, the soles of the feet are joined in order to constrain the
afflictive vital energies in their own place. The vajra\hyp{}fists are
placed on the ground to cut off the pathways of the afflictions. The
gaze is directed upwards to open the vision of primordial wisdom. With
the sa\d{m}bhogak\a={a}ya posture, pointing the soles outward causes
the vital energies to flow easily. Pressing your knees against your
chest balances the heat and cold elements of the body. Pointing your
knees and elbows at the ground blocks the impure apertures. Supporting
your cheeks with your palms balances bliss and emptiness. By directing
your gaze straight in front of you, primordial wisdom settles in its
own luminosity. With the nirm\a={a}\d{n}ak\a={a}ya posture, the soles
of your feet press on the air ma\d{n}\d{d}ala in order to suppress the
power of the karmic vital energies. By pressing together the fire
ma\d{n}\d{d}ala of the thighs and the fire ma\d{n}\d{d}ala of the
belly, the impure vital energies of sa\d{m}s\={a}ra are extinguished
right where they are. By pressing together the water ma\d{n}\d{d}ala
of the knees and the fire ma\d{n}\d{d}ala of the palms, the heat and
cold elements of the body are balanced. By pressing together the fire
ma\d{n}\d{d}ala of the palms and the fire ma\d{n}\d{d}ala of the
armpits, cold disorders are dispelled. By pressing together the water
ma\d{n}\d{d}ala of the backs of the hands and the water
ma\d{n}\d{d}ala of the throat, heat disorders are dispelled. By gazing
downwards, the eye of omniscience is opened. Even if you look straight
ahead or turn your gaze upwards, the eye of omniscience is still
opened, so there is no difference. You may direct your gaze wherever
you find the greatest clarity.

Moreover, it is not necessary to use all three postures. Rather, you
may stay in any of the postures that facilitates the arising of the
clear light and that you find comfortable and suitable. If you like
variety, you may shift from one posture to the other and from time to
time apply yourself to other spiritual practices. If you want nothing
complicated, strive in meditation continuously throughout the day and
night. Those who can meditate only during the day and not through the
night should constantly practice throughout the day. The practice is
to have three special sessions during the night and to intermittently
train in the dying process.\footnote{This refers to the preceding
  practice of imagining the dying process as a preparation to the main
  practice of the Leap\hyp{}over.}

The important thing for the senses is that you look with eyes
partially open and that you do not suddenly open them wide, for that
will dull your vision, and it will prevent the appearances of the
clear light from manifesting; so do not rigidly fix your gaze. The
important thing for the vital energies is that you practice breathing
gently through your mouth through a little opening between your lips
and teeth; and pause for a moment with the breath exhaled. As for the
object of your gaze, in the beginning for about one month, during the
daytime direct your gaze one cubit\footnote{To protect your eyes, it
  may be better to direct the gaze about six feet away from the sun.}
from the sun; then the [practice during the] night will clear away any
problems of heat increasing in the eyes due to the
sun.\footnote{Although the daytime practice of gazing near the sun may
  impair one's vision, it is said that the nighttime practice of
  gazing at the moon may actually enhance one's vision. Most important
  is that one carefully examine whether one's practice is damaging
  one's eyesight and to alter the practice if that occurs.} In order
to achieve stability in the clear light, gaze at the moon in the same
way.

At night if you gaze at a flame, by looking above it with your eyes
half open, at first you will see nothing more than something like an
orange bale of hay. After awhile, the absolute nature will appear and
bindus will arise in the form of quivering lines. Finally, beautiful,
limpid visions of the absolute nature and bindus will appear clearly
and extensively. Remain with your body unmoving like a corpse in a
charnel ground; keep your voice silent, avoiding all articulation; and
do not exhale through your nose but slowly breathe through your mouth
without impeding or forcing it. That is the reliance upon the crucial
point of letting the channels and vital energies be, without retention
or manipulation. Remain without moving from the state in which
consciousness experientially emerges as the clear light, without the
mind being modified in any way. Wherever you are, by keeping the body
straight, all the channels and vital energies will be straight, and
once the mind has dissolved into empty awareness, you will be
stabilized in that state.

The explanation of the channels and bindus of the path according to
this y\a={a}na is called \emph{ati\-anu}, so you should come to know
them correctly. The mouth is the aperture through which coarse, impure
mental afflictions, vital energies, and mental states manifest, and
the nose is the subtle aperture for subtle afflictions, vital
energies, and mental states. Here is the way they move. In the lungs,
channels having the width of a straw of wheat are filled with that
which is called the exhaled and inhaled breaths. If they increase,
heat disorders arise, if they decrease, cold disorders occur, and if
the breath flows smoothly, there is a balance of the heat and cold
elements of the body. In one day, there are 21,600 breaths, which are
like mounts for mental ideation. Therefore, even though there are
profound methods for forcefully constraining the vital energies and
mind by retaining and manipulating the channels and breath, they may
be enormously obstructive and misleading.

The six kinds of lamps\footnote{The use of the term lamp (Tib. sgron
  me) is, of course, a metaphor, for the essential nature of these
  lamps is luminosity.} of the ground of the nature of existence are
the avenues by which primordial wisdom arises, and the eyes are the
evident apertures of primordial wisdom. The ears are the hidden
apertures of subtle primordial wisdom, and they are the pathways by
which consciousness apprehends appearances, so you train in their
sounds. Through the evident apertures of primordial wisdom, you train
in the clear light that illuminates the darkness. Dream appearances
are the avenues to the manifestation of stainless vision, and by
familiarizing yourself with the clear light, emanation, and
transformation, the appearances in the transitional process of
becoming can be emanated and transformed. From that you can emanate a
pristine nirm\a={a}\d{n}ak\a={a}ya buddha\hyp{}field and accustom
yourself to transforming the appearances of the intermediate state.

Here in order to experience the visions of the embodiments and
primordial wisdoms there are three kinds of lamps of the vessel. The
quintessence of the body is the citta lamp of the flesh at the heart,
the inside of which is soft white. This is called the lamp of the
channels, the quintessence of the channels, and hollow crystal kati
channel. It is a single channel, one\hyp{}eighth the width of a hair
of a horse's tail, with two branches that penetrate inside the heart
like the horns of a wild ox. They curve around the back of the ears
and come to the pupils of the eyes. Their root is the heart, their
trunk is the channels, and their fruit is the eyes.

The quintessence of the apertures is called the fluid Lasso lamp. That
consists of three kinds of lamps of the vessel. Although the three
kinds of lamps of the vessel are given three names, in reality they
refer to the same thing, like a root, trunk, and fruit. Thus, in the
context of the path, they are simply called the fluid lasso lamp.

As for the three kinds of lamps of the vital essence, the lamp of the
pristine absolute nature is the quintessence of the five outer
elements. The transformation of impurities into the five\hyp{}colored
lights of the empty essential nature of the quintessence is called the
absolute nature, and because of the purification of the reification of
impurities, it is called pristine. The element appearing as space,
transformed into the quintessence, is blue and light blue. The element
appearing as water, transformed into the quintessence, is white and
gray. The element appearing as fire, transformed into the
quintessence, is red and brown. The element appearing as earth,
transformed into the quintessence, is yellow, pale yellow, and dark
yellow. The element of air, transformed into the quintessence, appears
as green, tan, and light green.

At first, in whatever color the impure visions appear, when they are
transformed into the absolute nature, they still appear in that same
color. As for the visions of the absolute nature, at first they are of
the nature of such things as the sun, moon, and a flame, bearing all
five colors, filled with rainbow patterns of the absolute nature, like
brilliant brocade. This rainbow weave arises as horizontal
images. Beginners may achieve stability in this by gazing for a month
at the sun and a crystal during the daytime, at the moon during the
nighttime, and by gazing at a flame while indoors. At the beginning,
shimmering images arise, after awhile they become more stable, and
finally they remain motionless. At that time, look at a clear window,
dispense with the flaws of enjoying or not enjoying the beauty or lack
of beauty of the light images. Then a whitish blue emerges which is
not that of the external sky, but know that it is important to rest in
a state without attraction or aversion to its qualities.

To transform the five inner elements into quintessences, the element
of the quintessence of the mind is transformed into blue and it
appears as such; the element of the quintessence of the blood
transforms into the color white and appears as such; the element of
the quintessence of the flesh transforms into the color yellow and
appears as such; the element of the quintessence of warmth transforms
into the color red and appears as such; and the element of the
quintessence of the breath transforms into the color green, and
appears as such.

As for the lamp of the empty bindu, the five quintessences appear in
circular forms, so they are called bindus. Although they are
spherical, without corners, in your vision they appear like concentric
circles due to throwing a stone in a pond. The interior of the
so\hyp{}called hollow crystal kati channel is filled with the lights
of the five quintessences, and a form of an indestructible bindu is
present in that space. By gazing at that with the eye of wisdom, the
interior of that channel becomes evident and arises in the form of
outer appearances. Without grasping onto them, your own channels will
illuminate themselves. If you grasp onto the visions of the absolute
nature as being external and onto awareness as being internal, you
will fall into the error of dualistic grasping.

In the domain of that pristine, absolute nature, the lusters of
awareness called vajra\hyp{}strands appear like moving, floating
threads of gold. That is the initial phase. After awhile they appear
like pearls threaded on a string, and finally they emerge in the form
of full and half\hyp{}lattices. They are the basis from which the two
kinds of lamps of vital essence arise, called the lamp of
self\hyp{}arisen wisdom and the self\hyp{}knowing sugatagarbha.

The four lamps of the path of appearances are the fluid lasso lamp,
the lamp of the pristine absolute nature, the lamp of the empty bindu,
and the lamp of self\hyp{}arisen wisdom. The four lamps of the
contemplative path are combined in one. Know that synthesizing them,
then applying yourself to practice is of the utmost importance. If you
practice in that way, unlike the mentally constructed, vague
meditation as in the Breakthrough, the reality\hyp{}itself of the
clear light will directly appear to your senses, so they are called
the direct visions of reality\hyp{}itself. This is not like meditating
on substantial, human\hyp{}like deities that are strenuously conjured
up by the mind, as in the stage of generation. This alone is the
practical instruction for achieving stability in the great
experiential displays of the embodiments and primordial wisdoms,
thereby liberating the actual three embodiments within yourself. This
is superior to the ordinary kinds of transference involving the three
recognitions,\footnote{The three recognitions (Tib. 'du shes gsum) are
  recognizing the channels as the path, one's own consciousness as the
  traveler on the path, and a buddha\hyp{}field as one's destination.}
a path by which you visualize shapes and colors and propel yourself
aloft, as it were. This has the distinction of the great transference
by which you transform all appearances and mental states of
sa\d{m}s\={a}ra and nirv\a={a}\d{n}a into the absolute nature of
reality\hyp{}itself.

Due to continuously practicing single\hyp{}pointedly in that way, the
potency of the vase empowerment strikes the materiality of your body,
so that you have no wish to move your body; due to the potency of the
secret empowerment permeating your speech patterns, you have no wish
to speak; and due to the potency of the wisdom empowerment striking
your mental continuum, your attention remains wherever you place
it. This is real quiescence that is devoid of signs. Thus, since all
coarse and subtle ideation is calmed in the ocean of the original
ground, it is quiescent; and since awareness remains without
fluctuation in its own state, it is still.\footnote{This sentence
  gives an etymology of the Tibetan term zhi gnas (Skt. \'{s}amatha),
  translated here as meditative quiescence.}

By transforming appearances and mental states into displays of the
embodiments and primordial wisdoms, there is an exceptional vision of
the clear\hyp{}light appearances of reality\hyp{}itself, so this is
called insight.

From the impure state of sa\d{m}s\={a}ra, since you truly know the
reality\hyp{}itself of the Breakthrough, the nature of existence of
suchness, you see the truth of reality\hyp{}itself. Due to achieving a
great, unprecedented vision of reality\hyp{}itself, this is the Very
Joyful [ground]. With the first visions of the Leap\hyp{}over, you
come to the confidence of never returning to sa\d{m}s\={a}ra, so you
implicitly achieve the first ground of the s\a={u}tra path. On the
mantra path, all delusive appearances and mental states come to
maturation in the nature of the clear light, reality\hyp{}itself;
ignorance is transformed into awareness, and you implicitly achieve
the ground in which awareness holds its own ground. At this time, even
if you die and are interrupted [in your practice], you will be reborn
as a t\"ulku, and you will have embarked on the path of
liberation. The outer signs are that the appearances of the absolute
nature are majestic and stable, as if the curtain on them had been
opened; and bindus appear, ranging from the size of fish eyes to thumb
rings.

This is the way the experiential visions progress. Initially, vital
energy fills you inside from your heart up to your throat, or various
sorts of illnesses or disagreeable pains may occur. Randomly moving
throughout the exterior and interior of your body, staying in no one
place for long, these disturbances arise due to the potency of the
vital energy of primordial wisdom striking the ascending wind. After
awhile, they increase and your throat may become sore and blocked so
that food is obstructed and coughed out. You may lose your appetite,
have trouble breathing, and lose your voice. Then they increase yet
further, and disturbances arise due to the potency of the vital energy
of primordial wisdom striking the life\hyp{}sustaining wind. Then you
may experience mood swings from joy to sorrow and from desire to
hatred. Due to the potency of the vital energy of primordial wisdom
striking the descending wind, when the disturbances increase, urine
and excrement are blocked and cannot be excreted, and when the
disturbances subside, they are expelled constantly. Due to the potency
of the vital energy of primordial wisdom striking the pervasive wind,
when those disturbances increase, the body becomes swollen, and when
they decrease, all the flesh of your body withers as if it were
becoming a corpse. Due to the potency of the vital energy of
primordial wisdom striking the fire accompanying wind, when the
disturbances increase, sweat emerges from the body and great heat
arises; and when they subside, you get goose bumps, your complexion
deteriorates, and you shiver with cold.

Finally, all the winds combine and enter the channels and elements of
the body, and sharp pains arise in all the channels. Movements of the
winds permeate your whole body, inside and outside, giving rise to
various illnesses such as combined heat and cold disorders. The body
becomes swollen, boils and sores appear, dire illnesses arise,
medications and divinations go awry, bad omens appear, and individual
channels and joints become painful. Gout, rheumatoid arthritis, and
lymph disorders may arise, and you may become lame, blind, deaf, or
mute, and may pass out. Know that various random kinds of pains may
arise in the body.

You may engage in various kinds of behavior, acting coquettishly or
shamelessly, like someone afflicted with a disease. In short, know
fully well that due to the functions of the channels, winds, and
elements, these bodily pains will not be the same for everyone, so
there is no one criterion for recognizing them.

As for your speech, you may find yourself singing various songs and
melodies, babbling, speaking offensively, having your behavior not
conform to your speech, not living in accord with your words and
acting contrary to them, and speaking uncontrollably as if your words
were uttered by an insane person. Such speech is nonsensical and
random, So recognize this!

Like the noises made by a madman, your mind may ramble aimlessly,
without being able to remedy or alter it in any way. Due to the
disorders in your heart and life force channel, at times you may weep,
groan, sigh, exhale forcefully, or constantly want to be on the move,
without being able to remain in one place. Your environment may seem
so miserable that you do not want to stay where you are, and you may
constantly experience a wide range of confused emotions. So recognize
this! You may have various sorts of visions of gods and demons or
random sensations of hunger, thirst, heat, cold, and so on. These are
the outer signs of the appearances of the clear light.

At the beginning stage, remain motionlessly with your face completely
covered, and bindus from the size of the dots on dice up to the size
of thumb rings will appear. At times, the visions of the absolute
nature, together with the bindus, will not be evident, but the luster
of awareness will appear in forms called vajra\hyp{}strands. At times
the bindus of the absolute nature will not arise, then they will
fluctuate in size, and they will become unclear, no matter how much
you try. On occasion, the visions of the absolute nature will
repeatedly appear in the expanse of clear light in spherical forms of
five\hyp{}colored lights. Those are the criteria of familiarity with
the practice.

At this time, even if your life comes to an end, you will go straight
to a nirm\a={a}\d{n}ak\a={a}ya buddha\hyp{}field, with no intermediate
state. By gaining greater familiarity with this practice, the visions
of the absolute nature will appear resplendently like loosely woven
cloth, and they will appear in the sky in the form of dangling
lattices and half\hyp{}lattices. In the midst of the visions of the
absolute nature, all sorts of images may appear, such as stupas,
lotuses, white conches, wheels, vajras, jewels, swastikas, swords,
spear\hyp{}tips, images like stacks of books, and various forms of
letters and animals. Whatever appears, those are visions of the
absolute nature, so know that it is important not to mistake them for
bindus. Bindus will appear in round shapes, gradually growing from the
size of thumb rings to the size of cups and on to the size of round
shields.

At the beginning stage, the lights of awareness, called
vajra\hyp{}strands, no broader than a hair's width, radiant like the
sheen of gold, appear to move to and fro, never at rest, like hairs
moving in the breeze. Even as they stabilize a little bit, they become
clear and lustrous, temporarily wavy, and they slow down somewhat,
appearing like deer running across a mountainside. Then as you become
somewhat more accustomed to the practice, they appear like strung
pearls, and they slowly circle around the peripheries of the bindus of
the absolute nature, like bees circling flowers. Their clear and
lustrous appearance is an indication of the manifestation of
awareness. Their fine, wavy shapes indicate liberation due to the
channels, and their moving to and fro indicates liberation due to the
vital energies. Due to the qualities of purifying the bindus, the
presence of bindus on curves [of the strands] indicates that one will
be liberated. By the power of meditation, they appear in the forms of
lattices and half\hyp{}lattices, transparent like crystal, radiant
like gold, and like necklaces of medium\hyp{}sized strung
crystals. The criterion of having thoroughly familiarized oneself with
the practice is that they appear indeterminately, but they remain
stable, without moving or vibrating. In this case, the name of the
cause is also given to the result, and that is the vajra\hyp{}strand
luster of awareness. They are the luster of awareness, so they
gradually become as stable as awareness itself. They are not the real,
self\hyp{}arisen lamps of awareness and wisdom.

Once the beginner's phase has passed, the visions of the absolute
nature become beautiful, clear, and stable, and they take on various
divine forms. Although they may increase and decrease, before a single
inner sign has arisen, the appearances of such outer signs are
premature, like a dzaki flower that blooms out of season. So that does
not constitute progress in terms of experiential realization. Even
when the inner signs occur, the outer bindus of the absolute nature
may become indistinct. That happens to some people who have a dominant
water element, and the elements of their channels are such that the
development of their experiential realizations proceed like a slow
foot race. If that is mistakenly identified as progress in meditative
experience and as reaching consummate awareness, even if visions occur
that would seemingly indicate the extinction into reality\hyp{}itself,
in fact one has in no way gone beyond ordinary consciousness. Indeed,
one is proceeding in the opposite direction, contrary to the tantras,
so this is an enormous error. It is important to know this.

Even if muddled outer signs and vivid images are present, recognize
the importance of the emergence of the inner signs. Although
experiential visions may appear to your inner consciousness, if the
outer signs are unclear, that indicates that you will not be able to
gaze at the clear light for sustained periods, and there will be
obstacles. Know this as well. When experiential visions homogeneously
arise inwardly, the visions of the Breakthrough are aroused, causing
the meditative experiences of the Leap\hyp{}over to be
disrupted. However, when the visions have not matured into the clear
light, the potency of the clear light has not been perfected. If they
stop, the visions of light will not develop, and that indicates that
the eye of primordial wisdom has not entered the eye of
wisdom.\footnote{The eye of wisdom sees appearances of the absolute
  nature, of bindus, and vajra\hyp{}strands that arise simply by
  attending to visions of light, whereas during meditative equipoise
  the eye of primordial wisdom sees the displays of the absolute
  nature and of bindus as they increase, stabilize, and become
  continuous.} Therefore, you should constantly strive in the
practice.

When encountering that situation, some people develop their minds with
meditative experiences, then travel to many regions and finally
succumb to adversities. Consequently, they get stuck there and do not
achieve liberation. Some people encounter images of the bodies,
speech, and minds of buddhas which are actually apparitions of
m\a={a}ras, gods, and demons, and due to visions from the power [of
  progress in meditative experience], words of Dharma appear to them
as written letters, and they are consumed by the desire to write them
down. Out of lust, they consort with women, and consequently claim to
be treasure\hyp{}revealers. There are many such people who bring ruin
to themselves and others. Due to extrasensory perception and visions
in dreams, some people perceive good and bad things in themselves and
others, and they leave such things as hand\hyp{}prints in rock and
other objects. Signs may manifest due to the apparitions of gods and
demons, causing them to declare themselves to be siddhas. They then
take a consort and take control of those around them. Laying the
foundations for prestige and great deeds, they spend their whole lives
in constant, relentless striving. Those who spend their lives tricking
others with magic rituals to dispel obstacles and wander around
begging and seeking wealth without satiation are possessed by
m\a={a}ras and demons. Even if they become renunciates and gurus with
great followings, they are deludedly involved in the eight mundane
concerns and the negative conduct of m\a={a}ras.

Some people take meditative experiences to be illnesses and regard
conducive circumstances as demonic. When they receive medical
treatment and perform rituals, they become confused by all kinds of
divinations and diagnoses, and they become overwhelmed by
speculations. Upon noting bad dreams and evil omens, fantasies arise
even more forcefully, and those outer manifestations are apparitions
of gods and demons. Any of the \oldstylenums{404} kinds of illnesses
in the body, including disorders of the wind, bile, phlegm, and
combinations of them, are inner manifestations as bodily pain. If you
regard them as being truly existent, you fall into error, and you will
either die or deludedly fall under the influence of objective
adversities.

Some people go through various kinds of unbearable miseries and
ecstatic experiences, all of them arising as secret manifestations
called joys and sorrows. If you cling to them and reify them, you will
stray into error, and you will not attain liberation. Due to misery
and discomfort and pain in the life force in the heart, people sigh
and feel like weeping, and everything they see and feel seems to be of
the nature of suffering. Then they restlessly yearn to escape
somewhere where there will be no human intruders, and when they come
to such a place of solitude, they yearn for companionship and moving
around. Overcome by desires and cravings, they find they cannot remain
in solitude, and they scramble after anything that will bring them
pleasure. That is falling into error, so recognize this!

Frightened by suffering, your body, speech, and mind become agitated,
impelling you to become active, and that is a great mistake. Some
people become depressed at the miserable pain in the life force in
their hearts, and out of despair when they wander from one village to
another, this seems to help. Then when they come around to their own
homeland and so on, before many days have passed, uneasiness arises
again just as it did before, and they wish to be on the move
again. Such people wind up squandering their whole lives.

Some people's minds are filled with doubt and vacillation, doubting
that they can ever come to certainty, and they waste their lives by
repeatedly traveling to many lands. Some are carried away by viewing
their teacher's counsel as being wrong, and they fall into false
views. Others take their own meditative practice to be harmful, and
they constantly feel regret and wonder what to do. They think that if
they were to go to some other famous spiritual teacher, that might
help. Nowadays there is not a single spiritual teacher who is well
versed in the nature of this path, the manner in which meditative
experiences arise, and so forth. Thus, fearing that their reputation
will decline, they cannot admit they do not know and are not familiar
with those things. Some of them teach things that are their own mental
fabrications, then tell others that their meditation is wrong. Others
say, ``Your guru doesn't know how to teach, so you have been
proceeding on a false path. Do this instead...'' Teaching that their
own level of instruction is all you need, they heap praise upon
it. There are a great many who pompously declare that they can
transfer their realization to others, saying, ``I shall grant you my
realization, our minds will merge, and you will simultaneously perfect
all grounds and paths.'' If that were possible, the buddhas would have
transferred their state of realization to sentient beings, and
sa\d{m}s\={a}ra would already be empty. Specifically, if the minds of
all the buddhas' \'sr\a={a}vaka and pratyekabuddha disciples received
the buddhas' realizations by having their minds merged with the
buddhas', why would they be drawn far beyond the H\a={\i}nay\a={a}na?
Do not place credence in pretentious assertions about transferring
one's realization.

Some people think they have no craving for the eight mundane
concerns. Others who have not developed their minds in the slightest
become obsessed with various visions they experience. Those
spiritually blind people never critically examine the way they wander
about in delusion, then claim they have reached the state of the
extinction into reality\hyp{}itself and think their own delusions have
vanished. Accomplished scholars scorn such attitudes and demolish them
with their weapons of scriptural authority and logic. So individuals
who enter this path should be careful in this regard.

Even if you succeed in other Dharma practices, you will not achieve
the highest state of liberation in this very lifetime. Consequently,
m\a={a}ras will not be jealous or angry, so they will not create
obstacles for you. If you do come to the culmination of this path, you
may achieve liberation in one lifetime and with one body. In this case
the might of the terrifying Lord of M\a={a}ras is dredged up, the
m\a={a}ras are aroused to jealousy and aggression toward those
advancing towards the state of spiritual awakening, and they are sent
out to create obstacles. They then create problems and manifest
objective apparitions to lead people astray.

By practicing single\hyp{}pointedly, without succumbing to such
obstacles, the appearances of light increase, and as soon as you
settle in meditative equipoise, all appearances become totally
pervaded by light and bindus, with no intervals between them. Ordinary
phenomena that appear due to looking at impure phenomena with the
eyes, are seen with the eyes of the flesh. The appearances of the
absolute nature, of bindus, or of [vajra\hyp{}] strands arise simply
by attending to the visions of light. They are derivative of the
manifestation of wisdom, so they are said to be seen with the eye of
wisdom. During meditative equipoise, the displays of the absolute
nature and of bindus increase, stabilize, and become continuous, and
they are said to be illuminated by the eye of primordial wisdom.

The consciousness that manifests the visions of the clear light during
the initial phase is called the eye of wisdom. Wherever the eye of
primordial wisdom, free of dust, is directed, it illuminates whatever
it sees until the visions of the absolute nature, the bindus, clear
light, and divine embodiments are seen. Then due to the sharp pinnacle
of primordial wisdom, free of fluctuations in the clarity of the eye
of wisdom, all appearance, both while in meditative equipoise and
otherwise, transform into displays of light and rainbow bindus with
ever increasing clarity. In the end, appearances of earth and rock
vanish and dissolve into continuous, omnipresent displays of visions
of light. That is the criterion for have acquainted oneself with this
practice. Impurities have been transformed into the vital core, and
the vital core has been transformed into the five lights, and they
become manifest. That is the criterion for perfecting the power of
progress in meditative experience. At this juncture, the larger bindus
cover the sky and earth, while the smaller ones variously appear as
small as grains of mustard, and they appear in aggregates of
five. Within the visions of the absolute nature appear the doors,
roof, Dharma wheel, crowning parasol, strings of bells, and silk
hangings of a palace.

Individuals who embark on such a profound, swift path, who have the
fortune of combining their karma and prayers, will experience the
spherical images of the first phase even at the time of death. At that
time, they will expire in the nature of
nirm\a={a}\d{n}ak\a={a}ya. Finally, once the power [of their progress]
has been perfected and nothing appears other than the fivefold
aggregates of bindus, they will be liberated as sa\d{m}bhogak\a={a}ya,
without experiencing the intermediate state.

To present this in terms of the grounds and paths, when you come to
the state called progress in meditative experience on the path of the
Leap\hyp{}over, that is identified with the fifth ground of the
s\a={u}tra path called Difficult to Practice. These meditative
experiences are unbearably painful, and under their influence one
experiences craving and confusion. Therefore, when one comes to this
stage, since it is very difficult to follow the path to its
culmination, this is called Difficult to Practice. On the mantra path
all the appearances of birth and death in sa\d{m}s\={a}ra are cut off,
and one does not perish. This is the achievement of the state of a
vidy\a={a}dhara who has mastery over life itself.

Then the appearances of reaching consummate awareness, in which
awareness matures into its vital essence, are as follows.  The upper
portions of the divine embodiments appear in the midst of all the
fivefold aggregates of bindus, while the lower portions of their
bodies appear in forms of clouds of light. One half of their bodies
appears as if it were separated. At that time, by practicing
continuously, eventually entire divine embodiments will appear. The
white, solitary embodiment, replete with the ornaments of a
sa\d{m}bhogak\a={a}ya is Vairocana; the blue embodiment is
Vajrasattva; the yellow embodiment is Ratnasambhava; the red
embodiment is Amit\a={a}bha; and the green embodiment is
Amoghasiddhi. By continuing in constant practice, the embodiments
eventually appear in the form of male and female deities in union; and
they arise together with their entourages of the four male and female
bodhisattvas.

As a result of further, continuous practice, assemblies of the five
buddha classes appear in spacious, vast palaces, beautifully adorned
with all manner of ornaments, clothed in various silks, blazing with
rays of light, and adorned with bindus and minute spheres. By
familiarizing oneself with that more and more, volcanic mansions
appear that are inwardly constructed of three tiers of skulls, while
outwardly appearing as palaces. In their midst are ma\d{n}\d{d}alas of
ferocious blood\hyp{}drinkers. The deities and consorts are embraced
in union, and single male deities appear dressed in fresh elephant
skins, tied with belts of human skin, with lower garments of tiger
skins, each bearing weapons. They appear in all sizes from the larger
ones as vast as the sky, and the smaller ones as tiny as peas. The
entire universe appears to be filled and totally pervaded with rainbow
light and blazing fire. Objects as small as the head of a pin are
filled and illuminated with divine embodiments with all their
ornaments. That marks the perfection of the potency of reaching
consummate awareness.

The mark of one's speech at this point is that one's voice is soothing
and enchanting, like songs sung by the children of kumbhandhas. In
addition, various words of Dharma, legends, and knowledge of
linguistics, poetry, and composition naturally emerge. Appearances
arise as symbols and as scriptures, and the meaning of all oral
transmissions and practical instructions flows forth like the current
of a river. Words of melodious songs and so on inspire others'
perceptions of the world, and their minds are blessed.

The bodily signs are that your body vividly appears as mudr\a={a} of
the five buddha\hyp{}classes, like the appearance of a reflection in a
space of limpidity and luminosity, like a mirror\hyp{}image. The body
appears as a variety of reflections, as light as cotton, with no sense
of materiality. As an indication that bodily parasites have been
released into the clear light, it becomes free of lice and nits. White
hair turns dark, bright white new teeth grow in, and your muscles
become youthfully strong, and wrinkles clear away. The perceptions of
others shift simply by laying eyes on you, and they experience faith
and reverence. With the blazing forth of the warmth of primordial
wisdom, all thoughts of clothing are discarded, there is no longer any
sense of being cold, and you experience continual blissful
warmth. Casting off all thoughts of food, you can live for months and
years on the food of sam\a={a}dhi, the power of bliss and
emptiness. In each pore of your body are displayed unimaginable kinds
of abodes of sentient beings as well as buddha\hyp{}fields. That
indicates the achievement of mastery of miraculous emanations. With
your mastery of incarnation and emanation, you manifest an
inconceivable number of emanations in an unimaginable range of abodes
of sentient beings, and in a single instant you guide an inconceivable
number of sentient beings. You manifest an inconceivable number of
emanations in an unimaginable number of buddha\hyp{}fields, where you
make myriads of offerings, receive empowerments, and open up an
inconceivable number of avenues of sam\a={a}dhi. Such transformations
are displayed in your own and others' fields of experience, and you
send forth and disclose unimaginable emanations and miraculous
displays. Due to your pristine perception, appearances arise as
displays of buddha\hyp{}fields, and due to the pristine purity of the
mind\hyp{}itself, the universe arises as a display of divine
embodiments. Due to the pristine purity of your voice, sounds arise as
wheels of Dharma. Pure appearances pervasively arise as displays of
those three pristine purities, without even a speck of impure
appearances.

Once the union has been mastered, the many avenues of the impure cycle
of existence are purified, and can be united with the great bliss of
the absolute nature. Once liberation has been mastered, simply by
focusing your awareness you can bring to a state of liberation even
someone who has committed the five deeds of immediate
retribution. Once you reach mastery over the elements, you can
transform all things into gold, silver, and so on; and phenomena are
mastered such that you can transform water into fire, fire into water,
and so forth. Once you have mastered the \a={a}yatana\footnote{These
  \a={a}yatana presumably refer to the five ``signs,'' (Tib. mtshan
  ma, Skt. nimitta) that eventually arise due to meditating on the
  generic emblems of the five elements of earth, water, fire, air, and
  space. These practices are discussed in B. Alan Wallace, \emph{The
    Bridge of Quiescence: Experiencing Tibetan Buddhist Meditation}
  (Chicago: Open Court, 1998) in the chapter \emph{Quiescence in
    Theravada Buddhism.}} of the five generic emblems, you can
transform your body into the five elements, have your body take on the
shape of other creatures, and manifest yourself in various emanated
forms. Once you have mastered all stages of birth, dying, and aging,
when you want to transcend the three worlds, you will become awakened
in the absolute nature of the dharmak\a={a}ya, Samantabhadra. This
occasion is called awakening in the great openness above, without
reliance upon any of the virtues, vices, causes, or effects of all
your lifetimes. Without reliance upon the qualities of your karma or
the appearances of the intermediate state, all mental states and
appearance naturally awaken by themselves, like the dawn breaking in
the sky, and there is no death.

Reaching the state of consummate awareness on the path of the Great
Perfection means that you implicitly attain what is called the eighth
ground on the s\a={u}tra path, and you also implicitly achieve the
state of what is called a Mah\a={a}mudr\a={a} Vidy\a={a}dhara on the
path of generation.

Moreover, due to the inconceivable differences among people's
metabolisms and faculties, there is a corresponding, inconceivable
array of meditative experiences. Thus, they are not uniform and there
are no definite criteria for them. The foregoing descriptions are
simply metaphorical and symbolic. You must examine this with awareness
and ascertain that all appearances are of the nature of meditative
experience. So recognize this!

O Vidy\a={a}vajra, by practicing in that way, enthusiastic, courageous
individuals do not need to be concerned with such issues as the acuity
of their faculties, the quality of their karma, or their age, as is
the case on other paths. They are said to be of superior faculties
solely due to their enthusiasm and courage. Therefore, when those who
integrate Dharma with their lives, without becoming frustrated in
their meditative practice, experience the outer and inner appearances
of reaching consummate awareness, without confusing one for the other,
all phenomena will appear only as lustrous light, and no ordinary
appearances will ever arise again.

Finally, like a full moon, the appearances of all embodiments and
bindus gradually decrease in number. From your brain\footnote{The
  Tibetan term here (dung khang) literally means ``conch abode,'' but
  it refers to the brain.} a white mass of light, like a billowing
cloud emerges in the space in front of you. In its midst appears an
aggregate of five bindus, in the center of which appears Vairocana
with his consort, adorned with sa\d{m}bhogak\a={a}ya ornaments, and
surrounded by four similar deities in union. Above, below, and all
around those divine embodiments, vajra strands arise in the forms of
dangling lattices and half\hyp{}lattices, like rosaries of clear
crystals. Then blankets of light, white like the moon, emerge from the
hearts of those embodiments and penetrate down into the point between
your eyebrows. For seven or five days those blankets of light appear
as ornamental bindus stacked up like upside\hyp{}down conch
bowls. Finally, they dissolve into the point between your eyebrows,
transforming your body into a mass of light. You thereby receive the
immutable body\hyp{}vajra empowerment.

At this point, even if you die, with no intermediate state, you will
experience the central buddha\hyp{}field called Ghanavyuha and achieve
stability. There the entire ground is composed of precious
crystals. It is so vast and all\hyp{}pervasively immense that it
rivals the dimensions of space itself. Its surface is smooth and even,
like the face of a mirror. When you step down, it gives way, and when
you lift up, it rebounds. As the soles of your feet touch the surface
of the ground, the primordial wisdom of bliss and emptiness blazes
forth. Clouds of delicious aromas spread forth from hills covered with
medicinal plants, and the whole ground is completely covered with
brilliant lotuses of various colors. The sky is criss\hyp{}crossed
with lattice patterns of rainbow\hyp{}colored light, and forms of
rainbow canopies, parasols, victory banners, and pennants appear in
it. It is surrounded all around by a great moat of water bearing the
eight excellent qualities, and on its shores are pebbles of various
precious substances, turquoise meadows, and golden sand. All around
inside them are immense, majestic, lightly filled forests of
wish\hyp{}fulfilling trees.  In the groves around its ponds are flocks
of birds that are emanations of buddhas, white like the color of
conch, yellow like gold, red like coral, green like emerald, and blue
like lapis lazuli, as well as other colors such as black, tan, and
variegated. Their beautiful forms are pleasing to behold, and their
lovely voices proclaim words of the sublime Dharma, as they circle
around the ocean and alight on the wish\hyp{}fulfilling trees. In the
rivers are innumerable, lovely, enchantingly beautiful goddesses
emanated by daughters of the gods, n\a={a}gas, gandharvas, and
kinnaras, who are constantly making clouds of offerings and rendering
service.

In the center of that buddha\hyp{}field is a square palace with doors
on each of its four sides, produced by the natural appearance of
primordial wisdom. Its east side is composed of crystal, its south
side is composed of gold, its west side is composed of ruby, and its
north side is composed of emerald. Its roof is of lapis lazuli, and
its exterior and interior are spacious and luminous. Its whole floor
inside is made of precious rainbow crystals. When the light of the sun
and moon streams through its windows, the floor becomes covered with
rainbow light and bindus. Jewel lattices and half\hyp{}lattices hang
from its walls, and parasols, victory banners, pennants, and silk
ribbons flutter in the wind, giving rise to words of the sublime
Dharma, which are heard by herds of lovely deers. This vast, spacious
palace is beautifully adorned with thresholds, Dharma wheels, and top
ornaments of the sun and moon. It is exquisitely designed and is
replete with all ornaments.

In its center, adorned with rainbows and a mass of light is a broad,
high, jeweled throne supported by eight lions. On its lotus, sun, and
moon seat is the Bhagavan Vairocana, adorned with all the
sa\d{m}bhogak\a={a}ya ornaments, of the nature of the purified
aggregate of form, the embodiment of the primordial wisdom of the
absolute nature of reality. He is surrounded by an immeasurable
assembly of bodhisattvas on the tenth ground, and he is constantly
turning the wheel of Dharma. Recognize the importance of occasionally
bringing that buddha\hyp{}field to mind even while you are still on
the path.

O Vidy\a={a}vajra, when you who are following this path finally go
beyond that stage, red\hyp{}colored light emerges from your throat
spreading into the sky in front of you.  In the midst of that light a
fivefold aggregate of bindus arises, in the center of which appears
Amit\a={a}bha with his consort surrounded by the four male and female
bodhisattvas. Between them are red vajra\hyp{}strands in patterns of
lattices and half\hyp{}lattices, like rubies strung together. From the
hearts of those divine embodiments rays of red light appear which
strike your throat in the form of a string of bindus, like inverted
ruby bowls, and stack there. They appear to dissolve into your throat
for twenty\hyp{}one, seven, or five days. You thereby receive the
secret vajra empowerment of unceasing speech, and you achieve
confidence.

At this time, there is a discontinuity, a shift of appearances, and in
an instant the entire ground, vast and spacious, is composed of
rubies. When you step down, it gives way, and when you lift up, it
rebounds. The whole ground is completely covered with brilliant
lotuses with blossoms of various colors. The whole environment in all
directions is completely surrounded by inconceivable
buddha\hyp{}fields. There are naturally arising ambrosial ponds with
jewel pebbles, golden sands, turquoise meadows, wish\hyp{}fulfilling
trees, ambrosial springs, rainbow canopies, and various parasols,
victory banners, and pennants. Unimaginable offering goddesses are
constantly making offerings and rendering service, and in the center
of all this is a palace composed of rubies. Its inner walls are white
on the east, yellow on the south, red on the west, and green on the
north. Its roof is blue and blazes with blue light, and it is adorned
with all ornaments and fine attributes. In its center, is a lotus,
sun, and moon seat upon a jeweled throne supported by eight
peacocks. On it sits the Bhagavan Amit\a={a}bha, red in color, adorned
with all the sa\d{m}bhogak\a={a}ya ornaments and garments, of the
nature of the purified aggregate of recognition, the embodiment of the
primordial wisdom of discernment. He is turning the wheel of Dharma
for an immeasurable congregation of bodhisattvas on the tenth
ground. You are instantly transported into their midst, you stabilize
there, and achieve confidence in this state.

Then when you move beyond that point, blue light emerges from your
heart into the space in front of you like a billowing cloud, and in
its midst arises a five\hyp{}fold aggregate of vast, spacious blue
bindus. In their center is the principal deity Ak\d{s}obhya with his
consort surrounded by the four male and female bodhisattvas. Adorned
with all manner of ornaments, lattices and half\hyp{}lattices of blue
vajra\hyp{}strands arise in the spaces between them like garlands of
vairata. From the hearts of those divine embodiments blue light
billows forth, penetrating down into your own heart, where bindus
stack up in a column like inverted lapis lazuli bowls. They appear to
dissolve into your heart for ten days or longer. You thereby receive
the wisdom\hyp{}gnosis empowerment of the undeluded enlightened mind,
and you achieve confidence.

Even if there is an interruption at this time, with no intermediate
state, your appearances will shift, and you will experience the
southern buddha\hyp{}field of Abhirati, as vast as the absolute nature
itself. Its surface is smooth and limpid, like the face of a
mirror. Its color is blue like lapis lazuli and it is
criss\hyp{}crossed with lattice patterns of rainbow light. Verdant
hills of medicinal plants are beautifully adorned with various
flowers, wish\hyp{}fulfilling trees, lakes of water bearing the eight
fine attributes, golden sands, turquoise meadows, jewel pebbles, and
unimaginable goddesses making offerings, singing praise, and rendering
service.

In the midst of the sky and intervening space adorned with all manner
of lovely ornaments is a square palace with four doors. Its exterior
is blue in color like lapis lazuli and blazes with light. Its interior
is radiant and luminous with the colors of the five primordial
wisdoms. In its center is a jeweled throne supported by eight
elephants. Upon it is a lotus, sun, and moon seat on which sits the
dark blue Bhagavan Ak\d{s}obhya, adorned with all the
sa\d{m}bhogak\a={a}ya ornaments, of the nature of the purified
aggregate of consciousness, the embodiment of mirror like primordial
wisdom. One hand touches the earth, while the other is in the
mudr\a={a} of meditative equipoise.  Around him is assembled an
innumerable sa\.{n}gha of bodhisattvas, who are listening to the
Dharma from the Teacher while bowing their heads in respect. As your
appearances shift to this, you will attain liberation.

When you move beyond that point, yellow light emerges from your navel
into the space in front of you like a billowing cloud. Immediately,
the whole ground becomes luminous with yellow light like the color of
gold, and all other phenomena arise as displays of yellow light. In
the midst of that mass of light a five\hyp{}fold aggregate of large
bindus arises like a round shield, and in its center is Ratnasambhava
with his consort, surrounded by the four male and female
bodhisattvas. Lattices and half\hyp{}lattices of blue
vajra\hyp{}strands arise in the spaces between them like garlands of
amber. From the hearts of those divine embodiments yellow light
billows forth, penetrating down into your own navel. In that continuum
of light bindus appear to stack up in a column like inverted golden
bowls for five or seven days; and finally, they dissolve into you. You
thereby receive the primordial wisdom vajra empowerment, free of signs
and words, in which all excellent qualities are perfected.

Even if there is an interruption at this time, with no intermediate
state, your appearances will shift, and you will experience the
precious buddha\hyp{}field of Srimat, as vast as the absolute nature
itself, in which the whole ground is like the color of refined
gold. Its surface is smooth and even. It is filled with grassy hills
of medicinal plants blanketed with various flowers, ambrosial ponds,
purifying springs, and a myriad of clouds of offerings of such things
as wish fulfilling trees. In its center is a palace emanated by
primordial wisdom. Its exterior is like the color of precious gold,
and its interior bears the colors of the four kinds of activities from
the natural potency of the five primordial wisdoms. In its center is a
jeweled throne supported by eight supreme horses. Upon it is a lotus,
sun, and moon seat on which sits the Bhagavan Ratnasambhava, whose
body is adorned with the signs and symbols of enlightenment and with
all the sa\d{m}bhogak\a={a}ya ornaments, of the nature of the purified
aggregate of feeling, the embodiment of the primordial wisdom of
equality. He is surrounded by an immeasurable sa\.{n}gha of
bodhisattvas to whom he is constantly revealing the Dharma. With the
emergence of these appearances, you will achieve liberation.

When you move beyond that point, your body appears as five lights, and
from it emerges a mass of dark green light into the space in front of
you. In its midst appears a five\hyp{}fold aggregate of
five\hyp{}colored bindus of light, like a rhinoceros\hyp{}skin
shield. In it is the principal deity Amoghasiddhi with his consort,
surrounded by the four male and female bodhisattvas. The images of
their bodies are limpid, they are replete with all manner of
ornaments, and blaze with a magnificent mass of light. Everywhere
above and below them vajra\hyp{}strands appear in the forms of
lattices and half\hyp{}lattices, like turquoise garlands. As for the
upward and downward extensions, from the hearts of those divine
embodiments green light billows forth, like the color of emerald,
penetrating your genital region. In that continuum of light bindus
appear to form in a column like inverted turquoise bowls for ten days
or so; and when they are complete, they appear to dissolve into
you. You thereby receive an empowerment that grants you mastery over
the spontaneously present divine embodiments and displays of
primordial wisdom.

At this time, even if your appearances shift, you will experience the
buddha\hyp{}field of Karmaprapurana, in which the whole ground blazes
like the color of emerald. The entire environment is replete with all
manner of ornaments and fine characteristics, and in its center is a
palace, bearing all wonderful qualities. Its exterior is green like
the color of emerald, and its interior is of the clear, luminous
colors of the four kinds of activities from the natural potency of the
five primordial wisdoms. In its center is a jeweled throne supported
by eight pheasants. Upon it is a lotus, sun, and moon seat on which
sits the Bhagavan Amoghasiddhi, whose body, green in color, is adorned
with all the sa\d{m}bhogak\a={a}ya ornaments, of the nature of the
purified aggregate of compositional factors, the embodiment of the
primordial wisdom of accomplishment. He is surrounded by an
immeasurable assembly of bodhisattvas on the tenth ground for whom he
is constantly turning the wheel of Dharma. As your appearances shift
to this, you will achieve liberation.

When you receive the vajra\hyp{}empowerment of spontaneous, original
perfection and you go beyond the final purification of the visions of
meditative experience, all the ma\d{n}\d{d}alas of the
blood\hyp{}drinking deities in the skull mansions appear to
you. Rising up into the sky above, you let out a terrifying roar and
appear to dance in various ways, causing all realms of the universe to
tremble and shake, the great earth quakes with a great
roar. Consequently, the entire animate and inanimate universe
dissolves into the nature of light, and, with a wave of your hand,
your own body disappears into the realm of light.

At this time, you will achieve the four great confidences of
fearlessness. What are those four? Due to arriving at the ground of
your own being, the dharmak\a={a}ya, the nature of the original
protector, the primordial buddha, even if you have a vision of buddhas
filling the whole of space, you achieve the great confidence in which
there is not the slightest bit of faith or reverence for them. By
coming to spiritual awakening within yourself, in which you can be
neither benefited or harmed by any other causes or effects, you
achieve the great confidence in which there are no hopes for the
ripening of effects from their causes. By coming to the ground of your
own being, which is originally free of birth, cessation, and abiding,
even if you are surrounded by a thousand assassins bent on murdering
you, you achieve the great confidence that is devoid of even the
slightest trace of fear. By experiencing the state of the originally
pure, primordial protector, and coming to the state that is originally
free of delusion, you achieve the great confidence in which there is
no anxiety concerning sa\d{m}s\={a}ra or the miserable states of
existence.

Then the appearances of the absolute nature, the bindus, the divine
embodiments, and the buddha\hyp{}fields gradually vanish like the full
moon waning to the point that it disappears into the moonless
sky. Finally, awareness is awakened as the ground, and you come to the
nature of the dharmak\a={a}ya. The fundamental root of
self\hyp{}grasping is destroyed, and the mind of grasping is
extinguished. The ray of dualistic grasping is severed, thereby
extinguishing apprehended objects. Conceptualization involving
dualistic appearances is extinguished, so you expand into the even,
pervasive nature of the equal purity of sa\d{m}s\={a}ra and
nirv\a={a}\d{n}a. Your body becomes like a corpse left on a charnel
ground, so no fear arises even if you are surrounded by a thousand
assassins. Your speech becomes like an echo, reverberating back all
the sounds of others. Like a rainbow dissolving into the sky, your
mind expands into reality\hyp{}itself, free of conceptual elaboration,
a great, all\hyp{}pervasive state beyond all dimensions.

O Vidy\a={a}vajra, an individual who has extinguished the appearances
of all phenomena into the absolute nature of reality\hyp{}itself has
far exceeded the tenth ground of the s\a={u}tra path known as the
Cloud of Dharma. Such a one has implicitly reached the supreme ground
of a spontaneously present Vidy\a={a}dhara on the mantra path. Still
the most subtle of latent cognitive obscurations arise, and like the
illumination from a flash of lightning in the sky, on occasion your
body appears, for just the duration of a hand\hyp{}wave, as a body of
light in an expanse of light. Recognize that appearances and the mind
occassionally separate, and speech and words of Dharma are sometimes
uttered as they were previously. When this phase is completed from ten
days to ten months, the most subtle of cognitive obscurations vanish
into the absolute nature. This perfects the power of primordial wisdom
of knowing reality as it is, and you gain mastery of the originally
pure ground, the primordial dharmak\a={a}ya. By perfecting the power
of the primordial wisdom of seeing the full range of reality, you gain
mastery over the spontaneously present divine embodiments and the
displays of primordial wisdom. As the originally pure youthful vase
body, you are transformed into a totally perfected buddha, and you
become all\hyp{}pervasive.

Those having superior faculties are liberated as a great transference
embodiment, extending infinitely into the all\hyp{}pervasive
dharmak\a={a}ya, like water merging with water, or space merging with
space. Those having medium faculties attain buddhahood as a great
rainbow body, like a rainbow vanishing into the sky. When the ground
clear light arises, for those having inferior faculties the colors of
the rainbow spread forth from the absolute nature, and their material
bodies decrease in size until finally they vanish as rainbow bodies,
leaving not even a trace of their aggregates behind. That is called
the small rainbow body. When the ground clear light arises, the
material bodies of some people decrease in size for up to seven days,
then finally only the residue of their hair and nails is left
behind. The dissolution of the body into minute particles is called
the small transference. For those of superior faculties this
dissolution of the body into minute particles may occur even during
the Breakthrough.

O Vidy\a={a}vajra and the rest of you assembled disciples, listen and
consider this. These are the superior qualities of the spontaneously
present youthful vase body. The obscurations of ignorance are
dispelled in the absolute nature; ascending to the dharmak\a={a}ya,
beyond the total\hyp{}ground, lustrous primordial wisdom manifests,
and it transcends lustrous clarity. The primordial wisdom of seeing
the full range of reality manifests, and primordial wisdom transcends
the mind. Natural spiritual awakening within yourself surpasses
traveling to buddha\hyp{}fields. Free of all the extremes of
conceptual elaboration, it transcends causality of dependent
origination. Imbued with the eight freedoms,\footnote{(1) The freedom
  of the formed observation of form, (2) the freedom of the formless
  observation of form, (3) the freedom of beauty, (4) the freedom of
  limitless space, (5) the freedom of limitless consciousness, (6) the
  freedom of nothingness, (7) the freedom of the peak of mundane
  existence, and (8) the freedom of cessation. (Tsepak Rigzin,
  \emph{Tibetan\hyp{}English Dictionary of Buddhist Terminology}
  (Dharamsala, India: Library of Tibetan Works and Archives, 1986)
  p.~236.} it transcends all actions and their effects. The absolute
nature and primordial wisdom are equally pervasive, transcending
mundane existence. Perfectly complete buddhahood, imbued with nine
surpassing greatnesses is praised by all the jinas.

\end{document}
