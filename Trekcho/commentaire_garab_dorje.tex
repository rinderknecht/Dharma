%%-*-latex-*-

\documentclass[a4paper]{article}

% Language and fonts
%
\usepackage[utf8]{inputenc}
\usepackage[T1]{fontenc}
\usepackage[french]{babel}
\usepackage{hyphenat}

\title{Les Trois Maximes de Garab Dorjé et leur commentaire}
\author{Patrül Rinpoché}

\begin{document}

\maketitle

\begin{flushright}
\emph{Hommage au maître principal, bon et doté d'une inégalable
  compassion!}
\end{flushright}


\section{Les Trois Maximes}

Il s'agit de

\begin{enumerate}

  \item la confrontation immédiate (la vue);

  \item la décision en un seul point (la méditation);

  \item l'assurance dans la libération-destruction spontanée (la
        conduite).

\end{enumerate}

Point par point:

\begin{enumerate}

  \item Pendant \emph{\'{s}amatha}, pousser un \emph{pha\d{t}!}
        (regroupant méthode et discernement, soit \emph{up\={a}ya} et
        \emph{prajña}) fort et bref, qui détruit les mares
        stagnantes. L'on atteint alors une condition purement et
        totalement vide («~cela n'est rien, quelle
        stupéfaction!~»). C'est un état sans obstruction: «~cette
        expérience inobstruée est indicible~». «~Reconnais et préserve
        que cette indicible transparence est le corps de réalité (le
        \emph{rigpa} du \emph{dharmak\={a}ya}). Telle est l'explication de
        la maxime de Garab Dorjé relative à la vue.

  \item Quelque soient les états de l'esprit (tels que calme,
        agitation, colère etc.), si l'on reconnaît l'esprit, alors on
        reconnaît le \emph{dharmak\={a}ya} lui-même. Cette claire
        lumière qui nous avait été familière est comme la rencontre de
        la mère et du fils (ce dernier est la claire lumière de notre
        expérience, et la mère est la claire lumière du
        maître). Reposez-vous dans la condition de l'intelligence
        inexprimable. Dans cet état, s'il y a clarté, félicité,
        vacuité, alors détruisez l'attachement à ces expériences et
        continuez dans la clarté vide. La syllabe \emph{pha\d{t}!}
        tombe et détruit ces expériences comme la foudre: il n'y a
        plus alors de distinctions entre la méditation et la
        non-méditation (la méditation est ici l'expérience de l'espace
        et la non-méditation est voir toutes choses comme une illusion
        magique). On reste ensuite continuellement dans la condition
        indissociable. Cependant, tant que l'on n'a pas obtenu la
        stabilité en ayant rejeté les mondanités, attelez-vous à la
        méditation formelle. En effet, même si dans la vue du Dzogchen
        il n'y a pas de différences entre méditation et
        non-méditation, les débutants doivent s'appliquer à la
        méditation formelle au moins quatre fois par jour. En tout
        temps, en tout lieu, préservez la reconnaissance du
        \emph{dharmak\={a}ya}. Soyez pleinement convaincus qu'il n'y a
        rien au-delà, sans réticences, contemplez! Ainsi complètement
        résolu, tel est le point clé: la certitude que l'esprit est
        non né.

  \item Alors, tout ce qui est désir, souffrance et constructions
        mentales adventices ne sont pas continuées (l'image
        traditionnelle est celle des n{\oe}uds d'un serpent qui se
        délient seuls et sans effort). Il ne s'agit pas de rejeter les
        idées fictives, on les laisse telles quelles dans le flux
        constant de la reconnaissance. Il en va de même de tracés sur
        l'eau: ce qui de soi-même vient au jour, de soi-même disparaît
        sans traces. Il n'y a pas de solution de continuité, le tracé
        s'efface pendant qu'il apparaît, sans intervalle. Quelque
        soient les perceptions mentales qui surgissent, elles
        nourrissent l'intelligence vide et nue. Sans oubli, cela est
        naturellement et intrinsèquement pur. Quelle joie! Les
        expériences se présentent de la même façon à tous, mais le
        mode de libération est différent chez le pratiquant
        chevronné. Chez ce dernier, apparition et libération sont
        simultanées. Il ne craint pas de chuter dans les mondes
        inférieurs et n'espère pas devenir éveillé.

\end{enumerate}


\section{Le commentaire}

\noindent
Cet enseignement de Garab Dorjé se résume en trois points:

\begin{enumerate}

  \item la \emph{vue}: c'est la compréhension du mode d'être de la
        vacuité (\emph{dharmadh\={a}\-tu});

  \item la \emph{méditation}: c'est l'accoutumance à ce mode d'être de
        la vacuité;

  \item la \emph{conduite}: c'est la manière de maintenir cette
        accoutumance.

\end{enumerate}

\noindent
La manière de pratiquer ces trois points est la suivante:

\begin{enumerate}

  \item «~Le maître principal et sa lignée de transmission sont
        indissociables de notre esprit, tout y est inclus.~» Il est
        fait référence ici au corps de réalité non né. Dans la vaste
        sphère, qui est la nature de Bouddha exempte de prolifération
        discursive, tous les phénomènes sont compris: c'est la vue
        («~la vue est la vaste sphère de profusion~»).

  \item «~L'aspect cognitif de cette vue exempte de prolifération
        discursive est le discernement.~» Il est fait ici référence à
        la vue pénétrante (\emph{vipa\'{s}yan\={a}}) en tant que voie
        d'accès. Cette vacuité est indissociable de sa (re)cognition
        qui la tire au clair (aspect cognitif).

  \item La méditation \emph{\'{s}amatha} est l'aspect méthode. Si l'on
        joint méthode et discernement, mais aussi si l'on s'applique
        aux six perfections, alors on obtient la conduite. «~Ne pas
        dissocier vacuité et compassion, et la méditation est rayons
        de lumière de science et compassion.~»

\end{enumerate}

Pour celui qui pratique ainsi vue, méditation et conduite, il faudra
qu'il fréquente un lieu isolé car le monde est comme des épines qui le
distraient. Ayant en esprit totalement renoncé au monde, s'il a la
capacité supérieure de se concentrer uniquement sur le discernement et
la méthode compatissante, alors il se libérera dans un seul corps. Les
adeptes de capacité moyenne devront pratiquer \emph{\'{s}amatha} et
l'éminente inspection qu'est \emph{vipa\'{s}yan\={a}}. Sinon, la foi
et l'envie de pratiquer ainsi sont extraordinaires («~Si on ne fait
que tourner son visage vers cette pratique [...]~»); si les obstacles
sont intégrés à la voie, s'il n'y a pas trop d'espoir ni de crainte,
les vies futures verront la progression sur les terres des
bodhisattvas --- au cas où l'on ne renaît pas en
\emph{Sukh\={a}vat\={\i}}.

«~La vue est la vaste sphère de profusion~»: la présentation directe
de la vue est l'évanouissement soudain de l'esprit ordinaire. Les
fictions grossières et agitées, poursuivant les objets pour les
saisir, voilent la nature de l'esprit. Ainsi, la présentation du
maître serait inopérante («~notre visage ne se reflète pas sur de
l'eau agitée~»). Laissez donc votre esprit en liberté, tout à lui-même
et sans artifices, car il est la connaissance non amendée
(principielle) de la claire lumière spontanément présente --- on ne
comprendrait pas le mode d'être par des voies artificielles.

Même si, étant débutant, on s'applique à préserver la condition native
de l'esprit laissé tel quel, il n'est pas possible de dépasser un état
où il y a un attachement aux expériences de clarté, de félicité et
d'absence d'idées fictives. Dans cet état, sans appréhension et
tranquille, afin de briser la coquille de cet attachement et de mettre
en évidence le mode d'être d'intelligence nue et transparente
(non-obstruction de la clarté intérieure et extérieure), poussez le
cri \emph{pha\d{t}!} qui frappe l'esprit ordinaire. Avec ce cri, les
attachements sont pulvérisés et la libération se manifeste de façon
évidente --- \emph{hedewa!}

\begin{enumerate}

  \item la \emph{vue}: lorsqu'il n'existe pas d'objet appréhendé par
        l'esprit, c'est le \emph{dharmak\={a}ya}: «~deumeure tel quel,
        l'aspect de \emph{rigpa} est transparent (inobstrué)~». Il est
        important d'être d'abord confronté à la vue, sinon il n'y
        aurait pas de méditation, qui est accoutumance à la vue.

  \item la \emph{méditation}: c'est l'abandon dans l'état naturel (non
        né) incessant. Si l'esprit s'agite, les pensées sont la propre
        expression de la connaissance non amendée (l'analogie
        traditionnelle est la relation entre la fumée et le feu). Si,
        après la confrontation à notre état naturel, nous sombrons par
        manque de méditation dans les idées fictives, alors nous
        errerions dans la \emph{sa\d{m}s\={a}ra} pour de nombreuses
        vies. La claire lumière mère est la base à reconnaître, et la
        claire lumière fille, née de la pratique méditative, est la
        voie. Quand il y a conjonction indissociable de la base et de
        la voie, il y a rencontre de la mère et du fils. Si l'on
        préserve constamment par la méditation la claire lumière
        présentée par le maître, alors il n'y a plus lieu de bloquer
        les émotions perturbatrices et les idées fictives. Les
        attachements aux expériences de vacuité, clarté et félicité
        doivent être brisés. Quand ces expériences s'élèvent, ou la
        joie, alors la syllabe \emph{pha\d{t}!}, qui joint la méthode
        et le discernement, les pulvérisera comme s'abat la
        foudre. L'on fait alors son viatique de la non-distinction
        entre méditation et non-méditation. Il y a néanmoins un grand
        danger à retomber dans les idées fictives: il faut quatre
        sessions de méditation quotidiennes. On deviendra alors comme
        le vieillard qui observe les jeux des enfants.

\end{enumerate}


\end{document}
